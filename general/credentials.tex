\documentclass{article}
\usepackage[left=2cm,right=2cm,top=1cm,bottom=1cm]{geometry} 
\usepackage{graphicx} 
%\usepackage[section]{placeins} 
\usepackage[spanish]{babel}

%------------------------------ Constants ---------------------------------
\newcommand{\nombre}{Marco Flores}
\newcommand{\titulo}{Credenciales}
%----------------------- Custom commands ------------------------------
\author{}
\title{\textbf{\Huge\titulo}}
\date{}

\begin{document}
\maketitle
\section {Base de datos}
\begin{itemize}
        \item \textbf{ENGINE:} PostgreSQL
        \item \textbf{HOST:} synctruck-database-instance-1.cyusxcvzomzn.us-east-2.rds.amazonaws.com
        \item \textbf{PORT:} 5432
        \item \textbf{USERNAME:} marcoflores110
        \item \textbf{PASSWORD:} Dream79135
        \item \textbf{DATABASE:} synctruck
\end{itemize}
\section {Cache}
\begin{itemize}
        \item \textbf{ENGINE:} Redis
        \item \textbf{VERSION:} 6.0.5
        \item \textbf{REGION:} us-east-2
        \item \textbf{ARN:} arn:aws:elasticache:us-east-2:588957065232:cluster:sync
        \item \textbf{HOST:} synctruck-cache.tbc11t.ng.0001.use2.cache.amazonaws.com
        \item \textbf{PORT:} 6379
\end{itemize}
Al ser una instancia de Redis, esta se encuentra protegida a nivel de red. Esto significa
que la única forma de acceder a ella, aún con las credenciales correctas (en este caso
la dirección del host y el número de puerto son las únicas credenciales necesarias) no
será posible acceder a la base desde algún cliente externo al VPC de la base. En concecuencia
únicamente Lambdas pueden conectarse a ella y la instancia EC2 que se detalla más abajo.
La base puede ser apagada, degradada o eliminada directamente desde la consola de administrador
en aws.amazon.com de ser necesario.
\section {EC2}
\begin{itemize}
        \item \textbf{SO:} Ubuntu 20.0.4
        \item \textbf{USERNAME:} ubuntu
        \item \textbf{HOST:} ec2-18-216-4-246.us-east-2.compute.amazonaws.com
        \item \textbf{PORT:} 22
        \item \textbf{IDENTITY FILE:} server-key.pem
\end{itemize}

\section {Otros}
Como se ha expuesto en la documentación del sistema, gran
parte del mismo esta desplegado sobre varios servicios
distintos como lo son Lambda, ApiGateway o S3 por ejemplo.
Para acceder a estos recursos basta con acceder a la consola
de administración de Amazon. Puede utilizar el
siguiente link: aws.amazon.com

\end{document}
