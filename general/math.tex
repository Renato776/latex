\documentclass[titlepage]{article} %Use titlepage option to set title in a page of its own
\usepackage{blindtext}	%Use blind text to produce random text
\usepackage[left=2cm,right=2cm,top=2cm,bottom=2cm]{geometry} %Use geometry package to set custom margin sizes
\usepackage{graphicx} %Needed to insert images!
\usepackage[spanish]{babel}	%Use babel package to set language for the document.

\setcounter{tocdepth}{3} 	%Set to 5 if you need even further numbering.
% paragrahs and subparagraphs can be numbered too!
\setcounter{secnumdepth}{3}	%Same as above, set to 5 if further numbering is needed.
\usepackage{amsmath} 
\usepackage{titlesec}	%Use titlesec package to customize the formatting of titles
\titleformat{\section}[block]
  {\Large\bfseries\filcenter}	%Use Huge for even bigger headers
  {} %Add \thesection within the braces if you want to display header number.
  {1em}
  {}
%----------------------------- Constants ------------------------
\newcommand{\nombre}{Renato Flores}
\newcommand{\carnet}{201709244}
\newcommand{\universidad}{Universidad de San Carlos de Guatemala}
\newcommand{\catedratico}{Ing. José Saquimux}
\newcommand{\curso}{Matemática Aplicada 2}
\newcommand{\titulo}{Proyecto \#1}
% ---------------------- Custom comands -----------------------------
\newcommand*\rbreak{\par\noindent\linebreak} %My own custom linebreaks!

\begin{document}

\begin{titlepage} 
	\centering
	\includegraphics[width=0.15\textwidth]{/home/renato/latex/caratula/logo.png}\par\vspace{1cm}
	{\scshape\LARGE \universidad \par}
	\vspace{1cm}
	{\scshape\Large \curso \par}
	\vspace{1.5cm}
	{\huge\bfseries \titulo \par}
	\vspace{2cm}
	{\Large \nombre , \carnet \par}
	\vfill
	catedrático\par
	\catedratico

	\vfill

% Bottom of the page
	{\large \today\par}
\end{titlepage}

\clearpage
\section{Prueba de integracion entre Sympy \& Latex}
Este es un texto explicativo, abajo puede encontrar una
exprecion matematica que puede serle de interes. \\\\ 
$ P = \left[\begin{matrix}0 & header & header_{1} & header_{2}\\header & 0.6 & 0.4 & 0\\header_{2} & 1 & 0 & 0\\header_{3} & 0.7 & 0 & 0.3\end{matrix}\right] $
\end{document}
