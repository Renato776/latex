\documentclass{article}
\usepackage[left=2cm,right=2cm,top=2cm,bottom=2cm]{geometry} 
\usepackage{graphicx} 
%\usepackage[section]{placeins} 
\usepackage[spanish]{babel}

%------------------------------ Constants ---------------------------------
\newcommand{\nombre}{Renato Flores}
\newcommand{\carnet}{201709244}
\newcommand{\universidad}{USAC}
\newcommand{\catedratico}{Ing. Allan Morataya}
\newcommand{\curso}{Sistemas Operativos 1}
\newcommand{\titulo}{Pregunta 4, examen final}
%----------------------- Custom commands ------------------------------
\author{\nombre , \carnet}
\title{\textbf{\Huge\titulo}}

\graphicspath{{/home/renato/screenshots/}}
\begin{document}
\maketitle

\begin{figure}[h]
        \includegraphics[width=\textwidth,keepaspectratio]{tdc.png}
                 \caption{Transofrmada Discreta en Cosenos}
\end{figure}	
Como se puede observar, se utilizaron los comandos dct para calcular
la transformada discreta en cosenos con el parametro type = 2.
Para recuperar la señal original, se utiliza el mismo comando
pero con parametro type = 3 y se aplica sobre la nueva señal para 
recuperarla. 
\end{document}
