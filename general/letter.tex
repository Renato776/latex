\documentclass{letter} 
\usepackage[left=2cm,right=2cm,top=2cm,bottom=2cm]{geometry} 
\usepackage[spanish]{babel}	%
\usepackage{amsmath} 
% ---------------------- Custom comands -----------------------------
\newcommand*\rbreak{\par\noindent\linebreak} %My own custom linebreaks!

\begin{document}
\begin{letter}{Ing. Fernando Jose Alvarez Paz}
  \opening{Estimado Ingeniero,}
Comenzaré esta discusión citando el enunciado del problema. \\
``El estado de cada estudiante se observa al comienzo de cada 
trimestre por otoño. Por ejemplo, si un estudiante está en penúltimo
año al comienzo del trimestre del otoño actual, hay una probabilidad
de 80\% 
de que esté en el último año al comienzo del siguiente trimestre de 
otoño, 15\% de probabilidades de que aún sea estudiante de penúltimo
año y 5\% de que haya sido dado de baja. ( suponga que una vez 
que el estudiante abandona la escuela, nunca regresa ). Con este texto
responda. \\
Si un estudiante entra en State College como principiante
¿Cuántos años espera pasar como estudiante de State College? 
(solo escriba el número con 4 decimales), así como, la siguiente 
pregunta del exámen.''

Tras leer el enunciado, es imposible evitar notar algunos detalles que
hacen que el enunciado se sienta fuera de lugar.
  \begin{itemize} 
    \item En la primera pregunta, se hace referencia al nombre de 
      la institución (State College) Y se habla de ella como 
      si se nos hubiera introducido en el texto. Sin embargo, 
      en el enunciado nunca se hace referencia a la institución 
      sobre la cuál se realiza el estudio.
    \item En el enunciado se habla explicitamente sobre un penúltimo año
      y un último año. Sin 
      embargo, no otra información referente a la cantidad total de 
      años en la carrera es proveída. En las siguientes preguntas 
      se habla de un año principiante, por lo que ahora sabemos que 
      existen principiante, penúltimo y último. En ningún momento
      se habla o se hace referencia a la existencia de otros años en 
      la carrera.
    \item En el enunciado se nos provee explícitamente con 
      la distribución de probabilidad completa para el penúltimo año. 
      Sin embargo, no otra información explicita es proveída referente al 
      comportamiento de los otros años.
  \end{itemize}
  En vista de lo ambiguo del enunciado, 
  algunos solicitaron la cantidad de años a lo cual se respondió 
  que se usaran únicamente los datos dados en el enunciado. Demostrando 
  así que la carrera estudiada solo podría ser de 3 años. \\
  Habiendo aclarado esto, aún quedaban detalles que parecían faltar en el
  enunciado. Sin embargo, tras re-leer el enunciado varias veces
  finalmente encontré 
  la respuesta. Efectivamente, el enunciado estaba completo y era
  posible su solución empleado estrictamente y únicamente la información
  proveída en el texto. A continuación se presenta dicha solución.

  \begin{enumerate}
    \item Cuántos años existen en la carrera estudiada? \\
      En vista que el texto únicamente habla de 3 años, es imposible
      concluir otra cantidad.
    \item Cual es la distribución de probabilidad de cada año? \\
      A primera vista, podrá parecer que es imposible responder 
      esta pregunta únicamente con el texto planteado, pues este 
      únicamente nos habla de uno de los años. Sin embargo, si se estudia
      el texto mas detenidamente, se observa que las probabilidades 
      de cambio de estado de cada estudiante siguen la misma tendencia
      para todos los años. Esto se 
      hace notable al leer la palabra `` Por ejemplo '' la cual 
      alude al hecho de que todos los años se comportan de manera 
      identica, y por tanto proveer un ejemplo de uno de estos años
      es más que suficiente para que el lector generalice hacia los 
      siguientes años en la carrera.\\ 
      Tomando siempre la tendencia: 80\% avanzar de año. 5\% abandonar. 
      15\% repetir.
    \item Matriz de probabilidad \\
      $\left[\begin{matrix}  & principiante & penultimo & ultimo & graduados & deserciones
      \\principiante & 0.15 & 0.8 & 0 & 0 & 0.05\\penultimo & 0 & 0.15 & 0.8 & 0 & 0.05
      \\ultimo & 0 & 0 & 0.15 & 0.8 & 0.05\\graduados & 0 & 0 & 0 & 1 & 0
      \\deserciones & 0 & 0 & 0 & 0 & 1\end{matrix}\right]$
  \end{enumerate}
  El resto del procedimiento para responder las preguntas 1 y 2 es 
  facilmente obtenible a partir del punto 3. Obteniendo finalmente 
  las respuestas: 
  3.3217 años para la primer pregunta y una probabilidad de 
  0.8337 para la segunda. \\\\ 
  Ahora bien, fue para mí una gran sorpresa descubrir que estas 
  respuestas no habían sido aceptadas como correctas. Tras 
  consultar con algunos compañenos de clase que aparentemente 
  obtuvieron la respuesta correcta, descubrí lo que había sucedido.\\
  Tal parece que, el enunciado escrito en el parcial ya se nos había dado
  con anterioriadad. Este era uno de los problemas dados en la tarea
  preparatoria. Sin embargo, hay una gran diferencia respecto a estos 
  problemas. En el enunciado presentado en la tarea, se presentaba 
  una tabla que mostraba claramente y sin lugar a dudas el comportamiento
  de cada año ademas de listar todos los años que duraba la carrera
  en el estudio. \\
  Esta tabla sin embargo, fue omitida del enunciado del parcial. Con el
  objetivo de añadir claridad, a continuación se presenta dicha tabla.\\
  $\left[\begin{matrix}  & principiante & segundo & penultimo & ultimo & deserciones & graduados\\principiante & 0.1 & 0.8 & 0 & 0 & 0.1 & 0\\segundo & 0 & 0.1 & 0.85 & 0 & 0.05 & 0\\penultimo & 0 & 0 & 0.15 & 0.8 & 0.05 & 0\\ultimo & 0 & 0 & 0 & 0.1 & 0.05 & 0.85\\deserciones & 0 & 0 & 0 & 0 & 1 & 0\\graduados & 0 & 0 & 0 & 0 & 0 & 1\end{matrix}\right]$ \\\\
  Tras emplear esta matriz para el resto del procedimiento, 
  las respuestas son efectivamente las que se indican como correctas. \\
  Sin embargo, tras estudiar dicha matriz y el texto presentado en 
  el enunciado del parcial, es imposible que se haya podido llegar a 
  esa conclusion. \\ La prueba es la siguiente: \\
  \begin {itemize}
  \item En el enunciado del parcial, en ningún momento
    y de ninguna manera, ya sea directa o 
    indirectamente se habla de la existencia de un segundo año 
    ubicado entre principiante y penúltimo. No hay ninguna manera 
    lógica de poder concluir que dicho año existe y se encuentra en 
    esa posicion.
  \item El comportamiento de cada año varia ligeramente de manera
    arbitraria. Por ejemplo, el ultimo año tiene una probabilidad 
    del 85\% de avanzar (graduarse) pero el penultimo año 
    tiene una probabilidad del 80\% de avanzar. Este es únicamente 
    un ejemplo, puesto que todos los años varían su comportamiento
    ligeramente. Tambien es importante notar que no siguen 
    ningun patron en cuanto al cambio en su comportamiento. 
    Por tanto, es imposible que se hayan podido deducir dichas 
    distribuciones de probabilidad en base unicamente al texto del 
    enunciado.
  \end {itemize}
  En consecuencia, la única forma en que alguien pudo haber llegado
  a dicha matriz es copiandola directamente del enunciado de la tarea.
  Sin embargo, dado que el exámen no tenía la tabla presente en la 
  tarea, es evidente que son problemas distintos con respuestas 
  diferentes. \\\\
  Espero su respuesta y agradezco inmensamente su comprensión y 
  consideración. 
  \closing{Atentamente, Renato Flores}

\end{letter}
\end{document}
