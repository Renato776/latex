\documentclass{article}
\usepackage[left=2cm,right=2cm,top=2cm,bottom=2cm]{geometry} 
\usepackage{blindtext}
\usepackage{graphicx} 
\usepackage[section]{placeins} 
\usepackage[spanish]{babel}
\usepackage{listings} 
\usepackage{xcolor} 
\usepackage{pdfpages}
\usepackage{array} 
\usepackage{textcomp}
\setcounter{secnumdepth}{2}

%----------------------- Custom commands ------------------------------
\newcommand*\rbreak{\par\noindent\linebreak}
%------------------------------ Constants ---------------------------------
\newcommand{\nombre}{Renato Flores}
\newcommand{\carnet}{201709244}
\newcommand{\titulo}{Tarea \#7\rbreak Sobre conceptos de agilidad y Lean}

\author{\nombre , \carnet}
\title{\textbf{\Huge\titulo}}


\begin{document}
\maketitle
\section{Coloque el ciclo de vida del proyecto (Predictivo, Incremental, Iterativo, Ágil) de cada uno de los siguientes enunciados.}
\begin{tabular}{  | m{12cm} | m{3cm} | }
	\hline
	\parbox[t]{12cm}{
	 Combina ciclos iterativos e incrementales,
	 realizando iteraciones sobre un 
	 producto para obtener entregables intermedios
	 listos para usar en cada 
	 lanzamiento a lo largo del ciclo del proyecto.
	} & Ágil \\\hline
	 \parbox[t]{12cm} {
		 En las primeras iteraciones se entrega una funcionalidad básica y se va
		 agregando mayor funcionalidad al producto a medida que avanzan las
		 fases del proyecto.
	}	& Incremental \\\hline
	 \parbox[t]{12cm} {
En las primeras iteraciones se va construyendo un borrador del producto
final mediante el análisis-desarrollo-reflexión y en las fases sucesivas se va
agregando calidad al producto con más análisis-desarrollo-reflexión.
	}	& Iterativo \\\hline
	 \parbox[t]{12cm} {
		 Hasta que no finaliza o está avanzada la fase predecesora, no comienza su sucesora
	}	& Predictivo \\\hline
\end{tabular}

\section{ Coloque en cada uno de los siguientes cuadrantes
el ciclo de vida del proyecto (Predictivo,Incremental, Iterativo, Ágil) más recomendado.}
\begin{center}
\begin{tabular}{ | m{6cm} | m{6cm} |  }
	\hline
  Incremental & Agil \\\hline
  Predictivo & Iterativo \\\hline
\end{tabular}
\end{center}
\section{Complete los cuadros en blanco de la siguiente 
tabla con las palabras ``fijo'' o ``flexible'' }

\begin{center}
\begin{tabular}{ | c | c | c | c |  }
	\hline
	Ciclo de vida & Costo & Tiempo & Alcance \\\hline
	Predictivo & flexible & fijo & flexible \\
	Adaptivo & flexible & flexible & fijo \\\hline
\end{tabular}
\end{center}

\section{Unir con flechas los siguientes conceptos del Manifiesto Ágil}
\begin{itemize}
	\item Colaboración con el cliente \textrightarrow  sobre negociación contractual
	\item Software funcionando \textrightarrow sobre documentación extensiva
	\item Individuos e interacciones \textrightarrow sobre procesos y herramientas
	\item Respuesta ante el cambio \textrightarrow	sobre seguir un plan
\end{itemize}
\section{Marque con una cruz a qué filosofía corresponde 
cada uno de los siguientes principios}

\begin{tabular}{  | m{11cm} | m{2cm} | m{2cm} | }
	\hline
	Principio & Manifesto Ágil & Declaración de la Interdependencia \\\hline
	\parbox[t]{11cm}{
		Aceptamos que los requisitos cambien, incluso
		en etapas tardías del desarrollo.
	} & X &   \\\hline
	\parbox[t]{11cm}{
		Disparamos el rendimiento mediante la 
		responsabilidad común sobre los resultados
		y sobre la propia efectividad del equipo
	} &   & X \\\hline
	\parbox[t]{11cm}{
		Entregamos software funcional frecuentemente,
		entre dos semanas y dos meses.
	} & X &   \\\hline
	\parbox[t]{11cm}{
		Esperamos lo inesperado y lo gestionamos 
		mediante iteraciones, anticipación y 
		adaptación.
	} & X &   \\\hline
	\parbox[t]{11cm}{
		Incrementamos el retorno de la inversión 
		centrándonos en un flujo continuo de valor.
	} & X &   \\\hline
	\parbox[t]{11cm}{
		La simplicidad, o el arte de maximizar la
		cantidad de trabajo no realizado, es esencial.
	} &   & X \\\hline
	\parbox[t]{11cm}{
		Nuestra mayor prioridad es satisfacer 
		al cliente mediante la entrega temprana y 
		continua de valor.
	} & X &   \\\hline
\end{tabular}

\section{Coloque del 1 a 5, según el orden cronológico del 
Marco Ágil propuesto por Jim Highsmith}

\begin{center}
\begin{tabular}{ | c | c | }
	\hline
	\# & Fases \\\hline
	3 & Adaptacion (Adapt) \\
	5 & Cierre (Close) \\
	2 & Especulacion (Speculate) \\
	4 & Exploracion (Explore) \\
	1 & Visualizacion (Envision) \\\hline
\end{tabular}
\end{center}

\section{Complete en qué fase del Marco Ágil de Jim Highsmith se 
realiza cada una de las siguientes actividades.}

\begin{center}
\begin{tabular}{ | c | c | }
	\hline
	Fase & Actividad \\\hline
	Explorar & Aprender e incorporar el aprendizaje al siguiente equipo de proyecto \\
	Especular & Crear una declaración del alcance del producto y sus criterios de éxito \\
	Visualizar & Desarrollar un plan de lanzamiento, hitos e iteraciones\\
	Cerrar & Entregar funcionalidades en un corto periodo de tiempo\\
	Adaptar & Realizar correcciones, incorporar y retener las lecciones aprendidas\\\hline

\end{tabular}
\end{center}

\section{¿Cuáles son las 6 capas de la cebolla de planificación ágil propuestas por Mike Cohn?}

\begin{center}
\begin{tabular}{ | c | c | }
	\hline
	\# & Capa \\\hline
	1 & Estrategica \\
	2 & Portafolio \\
	3 & Producto \\
	4 & Despliegue \\
	5 & Iteracion\\
	6 & Diaria \\\hline
\end{tabular}
\end{center}

\section{Complete las palabras que faltan en cada uno de los 
principios del pensamiento Lean}

\begin{enumerate}
	\item Especificar con precision el \textbf{valor} 
		de cada proyecto
	\item Definir el \textbf{flujo o cadena} de valor del proyecto
	\item Permitir que el valor \textbf{fluya continuamente} 
		sin interrupciones
	\item Permitir que el \textbf{cliente} participe en 
		la identificacion del ``valor''
	\item Buscar de manera \textbf{constante} la perfeccion
\end{enumerate}

\section{Nombre 3 de los 7 desechos más comunes que hay que 
eliminar del flujo de valor}
	\begin{itemize}
		\item Sobreproducción
		\item Transporte
		\item Sobre procesamiento
	\end{itemize}
	
\section{En base a la filosofía Lean, complete el párrafo con 
alguna de las siguientes frases: ``Lo antes posible''
/ ``Lo más tarde posible '' / ``En el último momento responsable''}

Para no entorpecer el flujo de valor con demoras, deberíamos decidir
\textbf{en el último momento responsable} punto óptimo de
la compensación entre el tiempo disponible para una 
decisión y la necesidad de desarrollar un producto.
\end{document}
