\documentclass{article}
\usepackage[left=2cm,right=2cm,top=2cm,bottom=2cm]{geometry} 
\usepackage{graphicx} 
\usepackage[spanish]{babel}
\usepackage{titlesec}
\setcounter{secnumdepth}{3}

%--------------------- Custom commands ------------------------------
\newcommand*\rbreak{\par\noindent\linebreak}
\newcommand*\casoheadera[2]{
	\textbf{#1} & 
	\parbox[t]{12cm}{#2} \\\hline
}
\newcommand\casoheaderb[2]{
	\multicolumn{2}{|l|}{
		\parbox[t]{15cm}{ \textbf{#1} \\ #2 \\ } 	
	} \\\hline
}
%------------------------ Constants ---------------------------------
\newcommand{\nombre}{Renato Josué Flores Pérez}
\newcommand{\carnet}{201709244}
\newcommand{\titulo}{Exámen Final}
\graphicspath{{/home/renato/latex/general/scrum/master-develop/images/}}

%------------------------ Title formating --------------------------

\author{\nombre , \carnet}
\title{\textbf{\Huge\titulo}}

%------------------------- Document --------------------------------
\begin{document}
\maketitle
\textbf{\huge{Parte A: Caso Netflix}}\rbreak
Conforme a la situación descrita en el caso y bajo
la óptica de la ingeniería de Software, analice y
diseñe el conjunto de aplicaciones de Software
que Netflix cesita, incluyendo la base de 
suscriptores, la gestión de inventarios, el software
por recomendación y los pedidos de los
clientes, para la transición hacia el modelo actual de
distribución de contenidos, cuyo servicio principal
es la distribución de contenidos audiovisuales
a través de una plataforma en línea o 
servicio de video bajo de manda por streaming. Utilice
los diagramas UML que considere oportunos para
modelar su solución, de tal manera que quede 
suficientemente claro todo el proceso de desarrollo
del software.
\section{Tomando en cuenta la competencia y demanda de servicio,
      qué metodología de desarrollo de software adoptaría
      y por qué.}
\subsection{AUP}
Tomando en cunta la competencia, la demanda de servicio, y la 
posición actual de Netflix, la metodología de desarrollo de software 
que debería adoptarse es AUP (Agile Unified Process). \\
En vista de los antecedentes, realizar una transición 
desde un sistema de alquiler de vídeos en línea a un sistema de
consumo de video a solicitud (Streaming) es sin duda un cambio total
de paradigma. Este cambio claramente necesitará más que grandes 
esfuerzos de ingeniería de software para lograr su implementación
exitosa.  Este cambio también debe tomar en cuenta la resistencia
al cambio que muy seguramente surgirá entre el personal
administrativo, ejecutivo y financiero. Para lograr la transición
de manera exitosa, se debe prestar especial atención al análisis
y diseño de los sistemas de software a implementar, así como de 
asegurarse que todos los departamentos de la empresa entiendan
el cambio y manejen los nuevos conceptos, de manera que 
puedan fácilmente capacitarse para el uso de las nuevas
herramientas una vez estén listas. \\
Por esta razón, se ha escogido AUP. Netflix debe encontrar un 
balance entre agilidad y planeamiento, algo que AUP sin duda ofrece.
\subsection{Planificación}
La implementación de AUP como metodología de desarrollo de
software para la correcta transición de Netflix de su modelo
de negocio actual hacia un modelo de negocio basado en Streaming
debe ser cuidadosamente planificada. A continuación se detalla la
planificación para la implementación de AUP en el contexto específico
de este caso de estudio. Se detallan las 4 fases de AUP y como 
será llevado a cabo la implementación de cada una, tomando en cuenta
la problemática, los objetivos y los antecendentes de Netflix.
\subsubsection{Inicio}
La etapa de inicio será especialmente útil para introducir el 
cambio y presentar las primeras ideas para una posible solución, su 
arquitectura y un estimado de los recursos que se necesitarán.
Estos detalles serán presentados a todos los interesados del proyecto,
asegurándose que todos los departamentos pertinentes de la empresa
(financiero, administrativo, ejecutivo) posean itempo suficiente
para evaluar, entender y resolver sus dudas en cuanto al proyecto,
el nuevo modelo de negocios y su implementación. \\
Una vez terminada esta fase, el equipo estará mucho más familiarizado
con los objetivos del proyecto, los cambios que traerá, como las
distintas áreas de la empresa serán afectadas y una idea clara
de por donde comenzar su implementación.
\subsubsection{Elaboración}
Durante la etapa de Elaboración el equipo se encargará de 
llevar a cabo el despliegue inicial de la arquitectura propuesta en 
entornos de desarrollo. Se tendrá tiempo para verificar la 
arquitectura y su viabilidad en base a hechos y experiencias reales
de su desempeño tanto por parte del equipo de desarrollo como 
de los principales interesados. Se tendrá la oportunidad de replantear
la arquitectura y realistar ajustes o realizar un total rediseño
de ser necesario en base a la retroalimentación del cliente y los
resultados obtenidos de las pruebas de desempeño y estabilidad.
Nuevamente, debido a la naturaleza innovadora del servicio a
implementar, Netflix no puede darse el lujo de equivocarse al momento
de definir la Arquitectura y desplegar un sistema que produzca
resultados subóptimos para el usuario final. Netflix no puede perder
la crédibilidad y confianza que poseen con su base de subscriptores
actual. La arquitectura escogida durante esta fase debe ser escogida
en base a considerables esfuerzos de pruebas e investigación en los
campos de redes, sistemas distribuídos, clusters y seguridad.
\\
Una vez se tenga definida la Arquitectura a implementar y configurados
los entornos de desarrollo y producción, se podrá dar inicio a la 
siguiente etapa.
\subsubsection{Construcción}
Durante esta fase el equipo de desarrollo
se encargará de escribir e implementar las distintas herramientas
de Software que se planificaron. El usuario será capaz de proveer
retroalimentación en base a cada una, se vereficará su desempeño
y corroborán su alineamiento con los objetivos del proyecto, 
su alcance y los requerimientos planteados.
\subsubsection{Transición}
Finalmente, la etapa de transición. Se realizará el despliegue del
sistema en el entorno de producción oficial. Se llevarán a cabo las
primeras capacitaciones oficiales del sistema al personal de la
empresa y se comenzarán las estrategias de marketing y distribución
para dar a conocer el nuevo sistema al usuario final.
\subsection{Notas finales}
Debido a que AUP es un método ágil, se tiene la ventaja que 
una vez finalizada la cuarta etapa, se dará comienzo a la siguiente
iteración. Se revisará el alcance y los requerimientos del proyecto,
se realizarán modificaciones de ser necesario y comenzará el
desarrollo de la siguiente versión. AUP es especialmente útil ya que
se caracteríza por poseer un tiempo considerablemente largo en 
comparación con otras metodologías ágiles para la culminación 
éxitosa de su primer iteración, en contraste el tiempo necesario 
para culminar las iteraciones siguientes es significativamente menor.
Esta propiedad es especialmente útil para Netflix, ya que gozará 
de un tiempo suficiente para plantear, planificar, diseñar y ejecutar
la implementación del nuevo servicio con el debido cuidado y 
concentración. Asegurándose de esta forma que su primer versión sea tal
cual la necesita el cliente y se alínee adecuadamente a la visión de Netflix de 
su nueva estrategia y modelo de negocio. Debido al cambio 
inminente de paradigma para el público general, Netflix debe
asegurarse que este cambio sea adoptado de la forma más gentil posible
y que los usuarios finales se sientan lo más cómodos posible ante
el cambio desde su primer impresión. Debido al contexto de Netflix,
esta primer impresión del usuario final hacia el nuevo sistema es
decisiva. En consecuencia, se justifica nuevamente la adopción de AUP
como metodología de desarrollo de software ideal para este nuevo proyecto.

\section{¿Qué momentos de crisis puede identificar y como los
prevendría?}
\subsection{Resistencia al cambio}
Uno de los riesgos más comúnes durante el proceso de implementación
de un proyecto de Software es la resistencia al cambio por parte
de algunos interesados o en ocasiones de departamentos enteros
de la empresa donde se esté llevando a cabo el proyecto. Debido
a la naturaleza innovadora del nuevo proyecto Netflix y al cambio
de paradigma que este implica, la resistencia al cambio será sin
duda un momento de crisis que deberá afrontarse. Es de vital
importancia no pasarlo por alto y estar debidamente preparados con
un plan de contingencia adecuado.
\subsubsection{Solución}
La mejor forma de solucionar este problema será por medio de 
charlas y seminarios informativos internos respecto al nuevo 
concepto de Streaming previos a la revelación del nuevo plan de
negocio. Las analogías con servicios de consumo de video similares 
como YouTube pueden ser de gran ayuda para las primeras
introducciones, listando las similitudes y 
haciendo especial énfasis en las diferencias. Además de las charlas,
seminarios y analogías, se recomienda que la explicación del nuevo
modelo de negocio sea llevado a cabo por nada más y nada menos
que el mismo Hastings, quién podrá responder todas las dudas
administrativas y financieras que podrían surgir luego de revelar
la ambisciosa propuesta. Finalmente, los diagramas de modelado
del negocio serán de gran ayuda no solamente para el equipo de 
desarrollo, si no también para los departamentos financieros,
ejecutivos y administrativos quienes podrán apreciar de una manera
más gráfica y natural el nuevo modelo de negocio. Es indispensable
que los diagramas de casos de uso y el de modelo conceptual sean 
cuidadosamente diseñados para expresar correctamente el nuevo 
modelo de negocios.

\subsection{Selección inadecuada de arquitectura}
Una selección inadecuada de arquitectura es un posible riesgo que
debe ser evitado a toda costa. Una equivación al momento de
seleccionar la arquitectura a implementar conlleva a una pérdida
de tiempo y recursos financieros altísima. Si bien es posible en 
algunos casos realizar modificaciones a la arquitectura de un 
sistema, estas son por lo general cambios mínimos, pero una 
reestructuración entera de la arquitectura es algo rara vez visto
en la práctica y comúnmente conlleva al fracaso total o parcial
del proyecto en cuestión. Este riesgo es especialmente probable
en el contexto de Netflix debido a que en aquella época las 
tecnologías de Streaming no habían madurado como ahora y aún 
se encontraban etapas experimentales o académicas. Era responsabilidad
de Netflix invertir en este campo y seleccionar una arquitectura 
adecuada por sí mismos.
\subsubsection{Solución}
La forma más adecuada de evitar este problema es no dudar en la
inversión adecuada de recursos en las ramas de investigación y 
redes. Deben contratarse investigadores altamente capacitados en 
las áreas de Ciencias de la Computación, Telecomunicaciones y 
Redes cuya único rol sea la investigación académica dedicada 
hacia las tecnologías de Streaming y que puedan proveer 
documentos de estudio y guías relevantes sobre los detalles a 
tomar en cuenta cuando 
se esté seleccionando la arquitectura de los sitemas a implementar,
prestando especial atención a temas de latencia, ruteo de paquetes,
estabilidad, disponibilidad, seguridad y costos. Estas guías podrán
ser utilizadas como base al momento de decidir la Arquitectura 
a implementar.

\subsection{Pérdida del horizonte}
Para llevar a cabo la implementación éxitosa del nuevo modelo
de negocios de Netflix, se planea la implementación de varios
sistemas de software que deberán comunicarse entre sí para formar
el sistema de Netflix en su conjunto. Debido a la cantidad
considerable de sistemas a implementar: Sistema de CDN, monitoreo,
administración, recomendaciones, suscripciones y cliente. Los
diferentes protocolos de comunicación a utilizar: AMPQ, gRPC,
HTTP, WS, RESP. Es posible que el equipo de desarrollo pierda 
el horizonte, es decir olviden los objetivos generales del proyecto
y la visión de funcionamiento de alto nivel del sistema de Netflix
como un todo.
\subsubsection{Solución}
La mejor forma de prevenir este riesgo es prestando especial 
atención a la redacción y mantenimiento de una documentación robusta
y cuidando la manera en que esta sea accesible en todo momento
tanto por el equipo de desarrollo como los interesados de la 
empresa. La documentación debe contener tanto detalle como sea 
necesario para aclarar y mostrar de forma gŕafica (diagramas) y 
escrita (descripciones y explicaciones) el funcionamiento del sistema
de Netflix como tal. Esta documentación debe ser redactada en un 
formato estándar para el modelado de Software, por lo que es
importante que sea redactada siguiendo las directrices UML y de esta
forma asegurar que todo el equipo de desarrollo pueda fácilmente 
entender e interpretar la documentación correctamente.

\clearpage\textbf{\huge{Parte B: Caso Occidental Engineering}}\rbreak
\section{¿Qué opinas sobre la decisión de Wayne?}
\section{¿Qué opinas sobre la decisión de Deborah?}
\section{¿Qué hubieras hecho tú en el caso de ser Wayne?}
\section{¿Y si hubieras estado en la situación de Deborah?}
\section{¿Te importa tanto la situación económica del hospital
        en este caso, como te preocupaba la de Occidental
        Engineering?}
\section{¿Cuál es la diferencia?}

	Si. Considero que Luis Aguilar es, efectivamente la persona
	idónea para elaborar y llevar a cabo el plan de rescate
	de MasterDevelop. Luis posee una alta trayectoria (más de
	quince años) en el área de innovación, por lo que una
	de sus principales características es su alta habilidad de
	adaptación y fácilidad de adoptar los cambios. Esto sin duda
	le permitirá adaptarse al nuevo modelo de negocios de 
	Softics y poder sortear con éxito el drástico recorte de 
	presupuesto que estaba su disposición en el pasado.
	\\\\
	Además de su habilidad para adaptarse al cambio, Luis
	estuvo presente en todo momento durante el desarrollo de
	MasterDevelop, desde su concepción como una idea, hasta
	su implementación en su primera versión y posteriores que
	situaron a MasterDevelop 3.0 como uno de los productos
	insignia de Softics. Esta experiencia sin duda es 
	indispensable para que Luis pueda evaluar que características
	del software mejorar, que características remover y que 
	se puede agregar para rescatar la viabilidad de 
	MasterDevelop como un producto competitivo en el mercado.
	\\\\
	Además de su experiencia en el desarrllo del sistema y 
	de su experiencia en el área académica, otra habilidad
	importantísima que indudablemente califica a Luis para
	tomar la responsabilidad de enunciar y llevar a cabo el 
	plan de rescate de MasterDevelop, es su conocimiento y 
	relaciones actuales con los clientes de Softics que ya han 
	adoptado a MasterDevelop como parte de sus herramientas. 
	Al estar Luis familiarizado con ellos, este podrá identificar
	de mejor manera que exactamente esperan los clientes obtener
	de MasterDevelop y de como poder mantener el producto de
	manera que sea posible su rentabilidad e innovación.

\section{¿Qué tipo de productos y servicios podrían 
integrarse a MasterDevelop para ser económicamente 
auto-sustentable dadas las restricciones que lo limitaban?}

\subsection{Diagramas de casos de uso}
\begin{figure}[ht]
	\centering
        \includegraphics[width=400px,keepaspectratio]{casos-uso-alto-nivel.png}
                 \caption{Diagrama de casos de uso de más alto nivel}
\end{figure}	
\clearpage
\begin{figure}[ht]
        \includegraphics[width=\textwidth,keepaspectratio]{casos-uso-general.png}
                 \caption{Diagrama de casos de uso de alto nivel}
\end{figure}	

\clearpage

\begin{figure}[ht]
	\centering
        \includegraphics[width=400px,keepaspectratio]{extended-1.png}
                 \caption{Diagrama de casos de uso expandido}
\end{figure}	

\begin{figure}[ht]
	\centering
        \includegraphics[width=400px,keepaspectratio]{extended-2.png}
                 \caption{Diagrama de casos de uso expandido}
\end{figure}	

\subsection{Casos de uso}
\subsubsection{Crear diccionarios}
\begin{tabular}{  | m{3cm} | m{12cm} | }
	\hline
	\casoheadera{Nombre}{Crear diccionarios}
	\casoheadera{Autor}{\nombre}
	\casoheadera{Fecha}{\today}
	\casoheaderb{Descripción}{
		Los diccionarios ayudan a la web semántica a dar significado 
		a las palabras encontradas en los textos de las vistas que 
		desean crear los usuarios. Ayudan a identificar sustantivos, verbos,
		artículos, adverbios y demás en base al contexto, la semántica y los
		metadatos del documento. Crear y definir diccionarios es de vital
		importancia para la web semántica por esta razón.

	}
	\casoheaderb{Actores}{
		Administrador, Instructor
	}
	\casoheaderb{Precondiciones}{
		Haber creado uno o más modelos.
	}
	\casoheaderb{Flujo Normal}{
		\begin{enumerate}
			\item Ingresar al panel de administración de MasterDevelop.
			\item En la barra de navegación, seleccionar Diccionarios.
			\item En el menú que se presenta, seleccionar crear Diccionario.
			\item Llenar el diccionario a mano (no recomendable) o seleccionar 
				importar. MasterDevelop soporta diccionarios simples en 
				formato XML, RDF, TOML, JSON y YAML.
		\end{enumerate}
	}
	\casoheaderb{Poscondiciones}{
		\\
		El diccionario será registrado y estará disponible para su posterior
		uso por parte del mecanismo de inferencias y la asignación automática 
		de metadatos.
	}
\end{tabular} 
% ''''''''''''''''''''''''''''''''''''''''''''''''''''''''''''''''''''''''''''''''''''
\subsubsection{Editar diccionarios}
\begin{tabular}{  | m{3cm} | m{12cm} | }
	\hline
	\casoheadera{Nombre}{Editar diccionarios}
	\casoheadera{Autor}{\nombre}
	\casoheadera{Fecha}{\today}
	\casoheaderb{Descripción}{
		Si bien es sencillo importar diccionarios, estos comúnmente
		son conjuntos simples de clave valor. Sin embargo, un 
		diccionario mucho más útil es aquel que aporta anotaciones,
		tipos, temas, sinónimos y relaciones entre las palabras. MasterDevelop
		soporta la asignación y aplicación de estas propiedades. Este
		caso de uso demuestra como pueden agregarse estas propiedades
		para enriquecer los diccionarios creados previamente de Master
		Develop
	}
	\casoheaderb{Actores}{
		Administrador, Instructor
	}
	\casoheaderb{Precondiciones}{
		Haber creado uno o más diccionarios.
	}
	\casoheaderb{Flujo Normal}{
		\begin{enumerate}
			\item Ingresar al panel de administración de MasterDevelop.
			\item En la barra de navegación, seleccionar Diccionarios.
			\item En el menú que se presenta, seleccionar editar Diccionario.
			\item En el drop down de diccionarios registrados, seleccionar
				el diccionario de interés.
			\item Utilizar la barra de búsqueda y los filtros para navegar
				con facilidad el diccionario. Por cada término de interés,
				seleccionar editar.
			\item Editar los datos del término seleccionado o agregar 
				la información pertinente.
		\end{enumerate}
	}
	\casoheaderb{Poscondiciones}{
		\\
		El diccionario será enriquecido con la información extra disponible 
		a cada término y como concecuencia, las inferencias y la asignación
		de metadatos será más acertada y rápida.
		}
\end{tabular} 
% ---------------------------------------------------------------------------------
\subsubsection{Definir relaciones}
\begin{tabular}{  | m{3cm} | m{12cm} | }
	\hline
	\casoheadera{Nombre}{Definir relaciones}
	\casoheadera{Autor}{\nombre}
	\casoheadera{Fecha}{\today}
	\casoheaderb{Descripción}{
		Una de las funciones más útiles de MasterDevelop
		y que sin duda es un factor decisivo al momento de
		posicionarlo como un producto innovador y útil, es la
		habilidad de poder definir relaciones entre vistas, 
		modelos y ontologías. De esta forma, la generación de 
		inferencias y sugerencias útiles para el usuario se vuelve
		una realidad. Ya que de lo contrario, el tiempo de búsqueda que 
		le llevaría al sistema para análizar todos los diccionarios, 
		ontologías y modelos para calcular sus inferencias sería demasiado
		prologado debido a su gran volumen y en consecuencia el usuario
		marcaría la prolongada espera como tediosa y MasterDevelop como
		poco útil y práctico.
	}
	\casoheaderb{Actores}{
		Administrador
	}
	\casoheaderb{Precondiciones}{
		\begin{enumerate}
			\item Estar registrado en el sistema como usuario administrador
			\item Haber definido una o más ontologías
			\item Haber definido uno o más diccionarios
			\item Haber definido uno o más modelos
			\item Haber definido una o más vistas
		\end{enumerate}
	}
	\casoheaderb{Flujo Normal}{
		\begin{enumerate}
			\item Ingresar a la página de administración de MasterDevelop.
			\item Seleccionar la opción de relaciones.
			\item El sistema mostrará un mapa conceptual gráfico que le ayudará
				al administrador a comprender las relaciones actualmente
				definidas en el proyecto.
			\item El usuario será capaz de crear, eliminar o modificar relaciones
				según le convenga con ayuda de una interfaz gráfica.
		\end{enumerate}
	}
	\casoheaderb{Poscondiciones}{
		\\
		La red de relaciones del sistema que puede crear el usuario a partir
		de esta herramienta le será de vital utilidad al sistema mientras este
		crece en robustez y dominio de datos. La red de relaciones le permitirá
		al sistema de inferencias saber donde priorizar sus búsquedas y le permitirá
		obtener resultados válidos con una gran velocidad.
	}
\end{tabular} 
%--------------------------------------------------------------------------------------
\subsubsection{Crear ontologías}
\begin{tabular}{  | m{3cm} | m{12cm} | }
	\hline
	\casoheadera{Nombre}{Crear Ontologías}
	\casoheadera{Autor}{\nombre}
	\casoheadera{Fecha}{\today}
	\casoheaderb{Descripción}{
		Si bien el Diccionario representa un conjunto de datos
		que poseen nombre, significado y tipo. Este es de poca 
		utilidad si no está bien definido qué es cada tipo y que 
		tipos se encuentran disponibles. Los tipos por defecto son 
		aquellos encontrados en el lenguaje natural que los desarrolladores
		de MasterDevelop consideraron oportunos para su implementación.
		Algunos ejemplos son ``Adverbios'', ``Artículos'', ``Pronombres'',
		``Sustantivos'', ``Verbos''. Sin embargo, estos tipos pueden ser
		extendidos o removidos por el usuario. Nuevos tipos pueden ser definidos
		y cargados al sistema. Al conjunto de tipos junto a su definición y
		propiedades MasterDevelop le llama Ontología. 
	}
	\casoheaderb{Actores}{
		Administrador, Instructor.
	}
	\casoheaderb{Precondiciones}{
		\begin{enumerate}
			\item Estar registrado en el sistema como usuario administrador.
		\end{enumerate}
	}
	\casoheaderb{Flujo Normal}{
		\begin{enumerate}
			\item Ingresar a la página de administración
			\item Seleccionar la opción Ontologías.
			\item Seleccionar la opción Crear.
			\item Seleccionar la opción Agregar Tipo.
			\item Llenar la información requerida por el Sistema
				para la definición de un Tipo.
		\end{enumerate}
	}
	\casoheaderb{Flujo Alternativo}{
		\\
		El sistema también soporta la carga de nuevos Tipos 
		de manera masiva. Estos deben ser cargados en un documento
		con formato XML que defina de manera concisa e inequivoca a cada
		tipo que será parte de la Ontología. Si alguno de los tipos
		a cargar viola las reglas de integridad de Tipos definidas
		por el sistema se alertará al usuario y se rechazará el archivo
		entero.
	}
	\casoheaderb{Poscondiciones}{
		\\
		El usuario será capaz de consultar, manipular y eliminar la Ontología
		creada. La ontología quedará registrada en el sistema para su posterior
		uso al momento de relacionarla con el resto de elementos del sistema.
	}
\end{tabular} 

\subsubsection{Asignar Metadatos}
\begin{tabular}{  | m{3cm} | m{12cm} | }
	\hline
	\casoheadera{Nombre}{Asignar Metadatos}
	\casoheadera{Autor}{\nombre}
	\casoheadera{Fecha}{\today}
	\casoheaderb{Descripción}{
		Los metadatos en el contexto de MasterDevelop son 
		pequeñas piezas de información que se le añaden 
		a un documento o vista que le permite a la red semántica
		identificar su nombre, tamaño, categoría, tema y demás 
		información relevante para que el sistema de inferencias
		sea capaz de relacionar el documento con los modelos,
		diccionarios u ontologías definidas en el sistema. Los
		metadatos pueden ser añadidos a un documento o vista de
		forma manual o automática. Esta última es una de las
		funcionalidades que hacen a MasterDevelop tan útil
		y reconocido.
	}
	\casoheaderb{Actores}{
		Diseñador, administrador, Instructor
	}
	\casoheaderb{Precondiciones}{
		\begin{enumerate}
			\item Estar registrado en el sistema.
			\item Haber creado una o más vistas.
		\end{enumerate}
	}
	\casoheaderb{Flujo Normal}{
		\begin{enumerate}
			\item Ingresar a la página de inicio del sistema.
			\item Seleccionar la opción Gestionar Vistas.
			\item Escoger la vista a la cuál se le desea asignar metadatos.
			\item Seleccionar la opción Asignar Metadato.
			\item Llenar el formulario que se le presentará con la información
				relevante para definir el metadato.
			\item Seleccionar guardar.
		\end{enumerate}
	}
	\casoheaderb{Flujo Alternativo}{
		La forma en que se asignan metadatos de manera automática 
		es bastante simple. El usuario simplemente debe enlazar la
		vista objetivo con un modelo en concreto. El sistema posteriormente
		se encargará de realizar un análisis de inferencia recursivo en busca
		de documentos similares relacionados con dicho modelo y creará 
		metadatos relevantes para el nuevo documento. La certeza de estos metadatos
		dependerá de la cantidad de documentos cargados al sistema y la robustez de 
		las relaciones dentro de la red semántica definidas por el usuario. Por esta razón
		esta opción no es recomendada cuando aún se están creando los primeros documentos
		de MasterDevelop. Sin embargo, es extremadamente útil cuando el
		sistema ha crecido y en este momento facilitará al usuario 
		la creación de nuevas vistas, volviendolo en cierto modo, auto-sustentable
		a partir de este punto.
	}
	\casoheaderb{Poscondiciones}{
		\\
		El usuario será capaz de recibir inferencias y sugerencias cuando visualice
		sus documentos por medio del navegador en base a los metadatos
		asociados a la vista en cuestión.
	}
\end{tabular} 

\subsubsection{Gestión de imágenes}
\begin{tabular}{  | m{3cm} | m{12cm} | }
	\hline
	\casoheadera{Nombre}{Gestión de imágenes}
	\casoheadera{Autor}{\nombre}
	\casoheadera{Fecha}{\today}
	\casoheaderb{Descripción}{
		El manejo de recursos multimedia es sin duda 
		clave para cualquier red web. MasterDevelop 
		por el momento soporta únicamente el manejo de 
		imágenes. Sin embargo, si el proyecto es rescatado 
		éxitosamente, se planea introducir soporte para 
		otros formatos multimedia en la versión 4.0
	}
	\casoheaderb{Actores}{
		Diseñador
	}
	\casoheaderb{Precondiciones}{
		\begin{enumerate}
			\item Haber creado una o más vistas
		\end{enumerate}
	}
	\casoheaderb{Flujo Normal}{
		\begin{enumerate}
			\item Ingresar a la página de inicio del sistema.
			\item Seleccionar Gestionar Vistas.
			\item Seleccionar la Vista a la cuál se le desea adjuntar imágenes.
			\item Seleccionar Editar.
			\item Seleccionar la opción Agregar Imágenes
			\item La interfaz de usuario cambiará a un formato que no le permitirá 
				editar texto, sin embargo creará un formato Drag and Drop sobre el 
				cuál el usuario será capaz de arrastrar y posicionar imágenes
				donde considere pertinente.
			\item Seleccionar Guardar
			\item La vista será actualizada e incorporará las imágenes que ha decidido 
				el usuario.
		\end{enumerate}
	}
	\casoheaderb{Poscondiciones}{
		\\
		El usuario podrá navegar y visualizar las imágenes en el navegador y 
		asímismo será capaz de editarlas o removerlas en caso de ser necesario.
	}
\end{tabular} 
\clearpage

\begin{figure}[ht]
	\centering
        \includegraphics[width=\textwidth,keepaspectratio]{modelo-conceptual.png}
                 \caption{Modelo conceptual, identificando los principales objetos de negocio}
\end{figure}	

\subsection{Resumen}
En resumen, los nuevos productos o servicios que deberían incomporarse, 
luego de haber modelado el negocio, reconocido los actores y haber
realizado el análisis de requerimientos son los siguientes.
\subsubsection{Crear un nuevo navegador}
Actualmente, MasterDevelop crea portales web de manera semi-automatica 
que son desplegables y consumidos sobre navegadores tradicionales. Sin
embargo, el desarrollo de un nuevo navegador, lo cual es mucho más
viable ahora debido a la precencia de éxitosos navegadores de código
abierto como Chromium o Firefox pueden servir de base o inspiración.
Este nuevo navegador tendría la ventaja que no necesitaría de un proceso
de compilación para que el portal creado por el usuario sea útil y 
consumible. Este nuevo navegador también ofrecería la ventaja
que el sistema de inferencias y sugerencias estaría implementado dentro
del navegador, agregando una experiencia mucho más enriquecedora 
e interactiva para el usuario.
\subsubsection{Añadir material didáctico}
Los conceptos de inferencia, sintaxis, ontologías, contexto, significado,
relaciones, búsqueda recursiva, entre otros serán una de las propiedades 
más atratactivas para las entidades académicas al evualuar el producto.
Por lo que la creación de tutoriales, ejemplos y documentación detallada
dentro del propio sistema y fácilmente accesibles para el usuario 
promoverá su nuevo enfoque académico además que disminuirá la necesidad
de personal de soporte técnico para la solución de dudas comúnes.
\subsubsection{Mejorar el sistema de inferencias}
La red semántica en que se basa actualmente MasterDevelop es sin duda
innovador y amigable. Sin embargo, la eficacia de estas inferencias 
puede ser mejorada al introducir conceptos como los metadatos y 
al definir peso para cada relación. La integración de metadatos y 
pesos en las relaciones le añadirá al mecanismo de inferencias la posibilidad
de calcular rutas de búsqueda mucho más eficaces para el recorrido del 
grafo de la red semántica. Permitiendo la implementación de algoritmos de 
recorrido basados en el Algoritmo de Dijkstra.

\section{¿Cómo resolverías el caso de mantener vivo a Master Develop?}
\subsection{Aplicar la filosofía Lean}
Debido a la naturaleza de la problemática que enfrenta Softics, es preciso 
adoptar la filosofía Lean para el desarrollo y mantenimiento del software.
Esto implica que el primer paso debe ser entender y tener bien definido
el significado de valor para cada uno de los clientes. En base a esta
información se debe crear un mapa de procesos que identifiquen todos los
procesos que están involucrados con el sistema. En base a este mapa y 
la definición de valor del cliente, se debe proceder a seleccionar cuidadosamente
los procesos más críticos e importantes, así como los procesos que no aportan 
valor al cliente en lo absoluto y deben ser eliminados.
\\\\
Una vez identificados y eliminados los desperdicios presentes en el
proyecto se obtendrá un recorte de gastos y ahorro considerable.
Lo cual se acopla apropiadamente a las nuevas políticas de
negocio de Softics.
\subsection{Cambiar el modelo de negocio}
Actualmente, Softics utiliza un modelo de licencia de código propietario
para la venta de MasterDevelop así como un plan de soporte técnico
y consultoría. Si bien el plan de soporte técnico representa una gran parte de
las ganacias de Softics en cuanto a MasterDevelop, esta estrategia dejará de 
ser viable en el futuro cercano debido al recorte de personal y en consecuencia
dificultad de proveer soporte técnico para nuevos clientes. Por esta razón,
se sugiere cambiar el modelo de negocio sobre el cuál se basa la venta y 
publicidad de MasterDevelop por un enfoque más académico. Es decir, 
dada la innovación inherente del producto y el alta estima hacia Softics
por parte del estado como empresa tecnológica nacional, deben aprovecharse
estas propiedades y vender una versión del software con un enfóque didáctico
y académico. Los posibles clientes deben ser universidades nacionales e 
institutos de enseñanza técnica. De esta forma, la necesidad de soporte técnico
dedicado podrá ser reducido en gran medida y reemplazado por modelos de subscripción
por parte de entidades educativas que pagarán por el software y clases de capacitación
de personal inicial. Una vez esté capacitado el personal educativo que elija nuestro
producto ellos se encargarán de enseñar su uso a sus propios alumnos y de proveer el 
soporte técnico durante los laboratiorios o actividades que se lleven acabo.
\\\\
Nuevamente, cabe recalcar la ventaja que posee MasterDevelop para ser distribuído como 
material académico, puesto que será una gran oportunidad para distintas organizaciones
educativas poder utilizar material nacional de alta calidad que les permita enseñar conceptos
clave sobre redes, hipertexto, redes semánticas, ontologías digitales, compiladores, 
interpretes, entre otros conceptos clave que se encargan del correcto funcionamiento de
MasterDevelop.
\subsection{Autosostenibilidad}
Una vez adoptado el nuevo modelo de negocio y luego de negociar éxitosamente algunos
acuerdos con diferentes instituciones educativas entonces deberá prestarse especial 
atención a las capacitaciones del personal que impartirá los cursos de interés. 
Este proceso podrá requerir de una inversión de recursos considerable, sin embargo estos
recursos financieros serán cobrados a las instituciones como parte de los costos
de distribución e instalación. 
\\
Una vez las capacitaciones hayan finalizado, el 
proyecto será autosustentable, ya que el personal interno de las insituciones estará
calificado y será responsable de dar el soporte técnico necesario a sus alumnos, 
provocando que Softics se preocupe solamente por cobrar las cuotas de uso mensuales o anuales.

\end{document}
