\documentclass{article}
\usepackage[left=4cm,right=4cm,top=3cm,bottom=3cm]{geometry} 
\usepackage{graphicx} 
%\usepackage[section]{placeins} 
\usepackage[spanish]{babel}

%------------------------------ Constants ---------------------------------
\newcommand{\nombre}{Renato Flores}
\newcommand{\carnet}{Análisis y diseño 1}
\newcommand{\titulo}{Trabajo de clase}
%----------------------- Custom commands ------------------------------
\author{\nombre , \carnet}
\title{\textbf{\Huge\titulo}}

\begin{document}
\maketitle
\section {Agile Unified Process - Lean software development - 
Dynamic Systems Development Method}
\subsection{Índice}
\begin{enumerate}
	\item \textbf{Agile Unified Process} \\\\
		Quick Overview
		\begin{enumerate}
			\item Historia
			\item Principios
			\item ¿Cómo ejecutar AUP?
			\item Ventajas
			\item Desventajas
			\item ¿Cómo decidir si AUP es apropiado
				para tu proyecto?
			\item Buenas prácticas y recomendaciones
		\end{enumerate}
	\item \textbf{Lean Software Development (LSD)} \\\\
		Quick Overview
		\begin{enumerate}
			\item Historia
			\item Los 7 principios de LSD
			\item ¿Cómo ejecutar LSD?
			\item Ventajas de utilizar LSD
			\item Desventajas de utilizar LSD
			\item ¿Cómo decidir si Lean es apropiado
				para tu proyecto?
			\item Buenas prácticas y recomendaciones
		\end{enumerate}
	\item \textbf{Dynamic Systems Development Method (DSDM)} \\\\
		Quick Overview
		\begin{enumerate}
			\item Historia
			\item Principios clave
			\item ¿Cómo ejecutar DSDM?
			\item Ventajas
			\item Desventajas
			\item ¿Cómo decidir si DSDM es apropiado
				para tu proyecto?
			\item Buenas prácticas y recomendaciones
		\end{enumerate}
	\item \textbf{Conclusiones}
	\item \textbf{Recomendaciones}
	\item \textbf{Bibliografía}
\end{enumerate}
\end{document}
