\documentclass{article}
\usepackage[left=2cm,right=2cm,top=2cm,bottom=2cm]{geometry} 
\usepackage{blindtext}
\usepackage{graphicx} 
\usepackage[section]{placeins} 
\usepackage[spanish]{babel}
\decimalpoint
\usepackage{listings} 
\usepackage{xcolor} 
\usepackage{pdfpages}
\setcounter{secnumdepth}{2}
\usepackage{apacite}
\usepackage{amsmath}
\usepackage{array} 
\usepackage{textcomp}

%\usepackage{hyperref}
%\hypersetup{
%    colorlinks=true,
%    linkcolor=blue,
%    filecolor=magenta,      
%    urlcolor=cyan,
%}
%----------------------------- Constants ------------------------
\newcommand{\nombre}{Renato Flores}
\newcommand{\carnet}{201709244}
\newcommand{\universidad}{Universidad de San Carlos de Guatemala}
\newcommand{\catedratico}{Ing. Fernando Álvarez Paz }
\newcommand{\curso}{Investigación de Operaciones 2}
\newcommand{\titulo}{Teoría de Colas}
%----------------------------- Custom Commands -----------------
\newcommand*\rbreak{\par\noindent\linebreak}

\graphicspath{{/home/renato/screenshots/math/}}
\begin{document}
\begin{titlepage} 
	\centering
	\includegraphics[width=0.15\textwidth]{/home/renato/latex/caratula/logo.png}\par\vspace{1cm}
	{\scshape\LARGE \universidad \par}
	\vspace{1cm}
	{\scshape\Large \curso \par}
	\vspace{1.5cm}
	{\huge\bfseries \titulo \par}
	\vspace{2cm}
	{\Large \nombre , \carnet \par}
	\vfill
	catedrático\par
	\catedratico

	\vfill

% Bottom of the page
	{\large \today\par}
\end{titlepage}

\tableofcontents
\listoffigures
\clearpage
\section{Introducción}
Las colas son parte de la vida diaria. Todos esperamos en colas
para recibir servicios, estos pueden ser cosas simples como 
comprar tortillas, pagar una boleta en el banco, pagar en caja
de un supermecado, etc. Si bien es cierto que esto forma parte de
nuestra vida diaria y es casi imposible pasar un dia completo sin 
esperar en alguna cola por cualquier razón, también es cierto
que siempre nos resulta molesto tener que esperar cuando las colas 
se prolongan demasiado.

El tiempo que la población humana pierde esperando en colas es un 
factor importante tanto en la calidad de vida como en la eficiencia
de la economía y otros campos. Esto debido a que el tiempo
perdido esperando en colas es tiempo que podría emplearse 
en alguna otra actividad.

Es por esta razón, que muchos matemáticos e investigadores se han
dedicado a estudiar estos sistemas detenidamente, estudiando
su naturaleza, componentes, comportamiento y demás detalles 
de interés con el objetivo de predecir su compartamiento y en
consecuencia tomar las mejores decisiones para optimizar detalles
como tamaño esperado de la cola, tiempos de espera promedio, entre 
otros son de extrema importancia e interés. 

Los investigadores a lo largo del tiempo han denominado este estudio
como ``Teoría de Colas''.

Para cumplir los objetivos de este documento se presenta toda la información
teórica relevante para resolver el presente caso de estudio en el Marco Teórico.
El texto del enunciado junto con su resolución se presentan en el Marco Práctico.

\clearpage
\section{Objetivos}
Evaluar distintas propuestas de optimización para bajar el nivel de inventario 
en proceso en la manufactura de alas para avión.
\subsection{Específicos}
\begin{itemize}
	\item Evaluar el estado actual del sistema.
	\item Comparar el costo actual del sistema con el costo estimado por 
		cada propuesta.
	\item Reducir el costo total por hora de proceso en el sistema.
\end{itemize}
\clearpage
\section{Marco Teórico}
\subsection{Definiciones}
Comenzaremos por la base teórica para responder preguntas como: 
¿Cuál es la definición de una cola?, ¿Cuáles son los componentes de una cola?, 
entre otras.
\subsubsection{Cola}
Se define como el lugar donde los clientes esperan antes de recibir un servicio.
Esta puede ser finita o infinita. Esta puede ser física o abstracta.
\subsubsection{Cliente}
Objeto que tiene como objetivo recibir un servicio y salir. Si el servicio 
está ocupado por otro cliente, este debe esperar en una cola.
\subsubsection{Sistema de Colas}
Un sistema que involucra colas de cualquier tipo.
\subsubsection{Mecanismo de Servicio}
Una o más estaciones de servicio, cada una de ellas con uno o más 
canales de servicio paralelos denominados Servidores.
\subsubsection{Disciplina de la cola}
Se refiere al orden con que los clientes son escogidos de la cola
para brindarles servicio. Existen las disciplinas: Random, FIFO, LIFO, entre otras.
Por lo general la más utilizada es FIFO (First In First Out) y por tanto 
es un estándar que si no se especifíca la disciplina de la cola de manera
explícita es porque la cola utiliza esta disciplina. Toda la teoría presentada
en el resto de este documento se basa en colas de esta disciplina.
\subsubsection{Tasa media de llegadas}
Se refiere al número esperado de llegadas por unidad de tiempo. Este dato
se representa con la letra griega $\lambda$ (Lambda).
\subsubsection{Tasa media de servicio}
En todo sistema de colas, la estación de servicio debe tener asignado algún 
tiempo que tarda en proveer dicho servicio a un cliente.  La tasa media de
servicio indica la cantidad esperada de clientes que completan su servicio 
por unidad de tiempo. Se denota con la letra griega $\mu$ (Mu).
En varias ocasiones existen múltiples servidores paralelos por estación 
de servicio, si la tasa media de servicio por cada servidor es distinta,
es necesario entonces agregar un subíndice a cada $\mu$ para indicar
a que servidor pertenece dicha tasa. Sin embargo, si la tasa media
es la misma para todos los servidores, basta con utilizar $\mu$.
\subsubsection{Número de clientes en el sistema de colas en un determinado momento}
Suele denotarse por la letra $n$.
\subsubsection{Factor de Utilización}
La fracción esperada de tiempo que los servidores individuales están ocupados.
Se denota por la letra griega $\rho$ (Rho). Un factor de utilización 
mayor que 1 indica que el sistema está sobre cargado y muchas de las fórmulas 
y asunciones que se realizan en este documento no aplicarían. Es importante 
para cualquier sistema de colas que desee mantener una longitud de cola y un tiempo
de espera promedio relativamente razonables que posea un factor de utilización
menor que 1. Este factor se define como:
\begin{equation} 
	\label{eq9:rho}
	\rho = \frac{\lambda}{\mu s}
\end{equation}
\subsubsection{Número esperado de Clientes en el sistema}
Este valor incluye tanto los clientes en cola como los clientes recibiendo servicio.
Este valor se denota por $L_{s}$ o simplemente como $L$. Es una de las
variables fundamentales en la teoría de colas. Su definición es independiente
del tipo de sistema y sigue la siguiente fórmula:
\begin{equation}
	\label{eq1:lsDef}
	L = \sum_{n=0}^{\infty} n Pn
\end{equation}
Donde Pn es es la probabilidad de que hayan exactamente n clientes en el Sistema.
La forma de encontrar Pn puede ser en ocasiones compleja 
y por lo general depende del tipo de 
sistema. Si bien esta es la definición para $L$, en la práctica es muy raro que 
se utilice para encontrar su valor, ya que evaluar series de manera exacta a  mano
es súmamente complejo: \\
Primero se debe verificar que la Serie converja. Si la serie diverge, entonces
el modelo está mal planteado o hay un error en la forma de obtener Pn. 
Una vez se está seguro de la convergencia de la serie se debe verificar si 
se asemeja a una serie conocida en el mundo matemático (Serie Geométrica, Hipergeométrica, Trigonométrica, Exponencial, etc) de ser así, se pude proceder a expandir dicha
serie y evaluarla; de lo contrario se debe recurrir a técnicas de cálculo, 
desplazamiento y cambios de variable sobre series para intentar su expasión a mano, 
dicho procedimiento es en extremo complejo y en ocasiones imposible. Si esto falla, 
se puede intentar obtener un valor aproximado de la serie evaluandola en un 
valor cercano al infinito. Es por esta razón que rara vez se intenta encontrar $L$ por
definición. Muchos matemáticos e investigadores del campo de IO a lo largo de la
historia han estudiado modelos específicos de sistemas de colas y 
realizado este trabajo por nosotros proveyéndonos simplemente con la fórmula 
para su evaluación directa, obtenida de la expasión y simplificación de esta 
serie para cada modelo conocido.
\subsubsection{Longitud esperada de la cola}
Este valor no incluye a los clientes siendo atendidos, sino únicamente a aquellos 
clientes esperando en cola. Este valor siempre es menor que $L$ y en ocasiones
puede ser cercano a 0. Se denota por $L_{q}$ y su definición es similar a $L$. 
\begin{equation}
	\label{eq2:lqDef}
	L_{q} = \sum_{n=s}^{\infty} Pn \left(n - s\right)	
\end{equation}
\subsubsection{Tiempo de espera en el sistema}
El tiempo de espera es una variable aleatoria que indica el tiempo que pasa
un cliente en el sistema ya sea esperando o siendo atendido. 
Se denota por $W$ y al ser una variable aleatoria
es necesario conocer su distribución de probabilidad, media y varianza para ser
definida. Estos dependen de que tipo de modelo se estudia y su deducción formal
no es trivial. En la práctica no es de interés conocer la naturaleza y definición
formal de $W$ sino únicamente el tiempo de espera promedio 
denotado como $w_{s}$ o simplemente $w$ y se define como: $w = E\left(W\right) $
Es decir el valor esperado de $W$.
\subsubsection{Tiempo de espera en la cola}
Similar a $W$ pero excluye el tiempo empleado en ser atendido, es decir
que sólo toma en cuenta el tiempo que pasa un cliente en la cola. Esta 
también es una variable aleatoria y se denota como $W_{q}$. Análogamente,
el tiempo de espera en la cola promedio o esperado es: $w_{q} = E\left(W_{q}\right)$

\subsubsection{Relaciones entre $L, W, L_{q}, W_{q}$ }
John D.C. Little demstró que en un proceso de colas en estado estable, se cumple:
\begin{equation}
	\label{eq3:lw}
	L = \lambda W
\end{equation}
%De manera análoga, se cumple:
\begin{equation}
	\label{eq4:lwq}
	L_{q} = \lambda W_{q}
\end{equation}
%Finalmente, se demuestra que:
\begin{equation}
	\label{eq5:wwq}
	W =  W_{q} + \frac{1}{\mu}
\end{equation}
%nothing here.
\begin{equation}
	\label{eq6:llq}
	L =  L_{q} + \frac{\lambda}{\mu}
\end{equation}

Todas estas ecuaciones son únicamente validas si $\lambda$ y $\mu$ no dependen
de la cantidad de clientes en el sistema y $\mu$ es constante para todos los servidores 
paralelos de la estación de servicio. En caso que alguno de estos supuestos no 
sea verdadero, estas relaciones no aplican. También es importante mencionar que estas
relaciones son independientes del tipo de distrución de tiempo de servicio, el proceso
de llegadas y el número de servidores.

\subsection{Notación de Kendall}
Como se ha visto
hasta el momento, la definición de la teoría de colas es bastante general y 
no impone restricciones en cuanto al proceso de entrada, distribución de tiempos
de servicio, capacidad del sistema, cantidad de estaciones de servicio, etc. 
De acuerdo a la variación de estos parámetros se crea un modelo u otro. Debido 
entonces a la exageradamente gran cantidad de combinaciones posibles y al 
desorden con que estos se encontraban originalmente, el profesor David George Kendall
propuso una notación en 1953 que tuviera como objetivo ordenar e identificar rápidamente 
los modelos de colas. Su notación fué bien recibida y aún se usa hasta la fecha.
Esta notación tiene la siguiente forma: \\\\
\begin{Large} $ A/B/C/D/E/F $ \end{Large} \\\\
Donde:
\begin{itemize}
	\item \large{A: }
		Código que describe el proceso de llegada. Los códigos más usados son:
		\begin{itemize}
			\item \large{M: }
				Significa ``Markoviano'' 
				(la tasa de llegadas sigue una distribución de Poisson),
				significando una distribución exponencial para los 
				tiempos entre llegadas. Tiende a ser el proceso más
				estudiado y es común encontrarlo en la realidad.
			\item \large{D: }
				Significa Tiempos entre llegadas Deterministas, 
				es decir, no siguen un proceso probabilista a la
				hora de su determinación. Tiende a usarse cuando 
				estos son totalmente constantes. Su utilización 
				en la práctica es muy rara.
			\item \large{G: }
				Significa ``distribución General'' de los tiempos
				entre llegadas, o del régimen de llegadas.
			\item \large{$E_{k}$: }
				Significa ``distribución de Erlang'' con parámetro
				k.
		\end{itemize}
	\item \large{B: }
		Código que describe el proceso de servicio. Los códigos utilizados
		son los mismos que en A.
	\item \large{C: }
		Número de servidores paralelos en la estación de servicio, también 
		se suele referir a ella como cantidad de canales de servicio.
		Suele reemplazarse por un número o por la literal  $s$ para
		indicar que no se sabe la cantidad de canales, pero se sabe
		que es mayor a 1.
	\item \large{D: }
		Indica la capacidad del sistema, es decir el 
		número máximo de clientes permitidos en el sistema incluyendo
		los clientes en servicio.
		Cuando el número está al máximo, las llegadas siguientes son rechazadas.
		Este parámetro no es obligatorio y suele omitirse cuando es igual 
		a su valor predeterminado. Su valor predeterminado es $\infty$.
		Suele utilizarse la literal k para indicar que la capacidad del 
		sistema es finita pero se desconoce su valor.
	\item \large{E: }
		Indica la disciplina de la cola. Valores usuales son:
		\begin{itemize}
			\item \large{FIFO: } First In First Out (cola).
			\item \large{LIFO: } Last In First Out (pila).
			\item \large{RANDOM: } Los clientes se seleccionan aleatoriamente.
		\end{itemize}
		Por lo general la disciplina FIFO es la más utilizada y estudiada
		tanto en la teoría como en la práctica. Por esta razón 
		este parámetro tiene como valor predeterminado FIFO y de no 
		específicarse un valor diferente explícitamente se sobreentiende 
		que el modelo hace referencia a esta disciplina. Suele omitirse 
		en la notación de la mayoría de modelos por esta razón.
	\item \large{F: }
		El tamaño del origen de las llamadas, es decir el tamaño
		de la población desde donde los clientes vienen. Por lo general 
		esta suele tomarse como infinita tanto en la práctica como 
		en la teoría. Sin embargo, hay ocasiones en las cuales la población 
		no es infinita, en dicho caso es necesario indicar el tamaño de la
		población de manera explícita.
\end{itemize}
No es necesario indicar los 6 parámetros de manera explícita para identificar un modelo.
Únicamente los 3 primeros parámetros son obligatorios, si no se especifícan los 
otros 3, estos toman sus valores predeterminados.\\
En algunos textos, se suele indicar la disciplina de la cola
como el último parámetro, siendo el penúltimo parametro el tamaño de la población.
\subsubsection{Ejemplos}
\begin{itemize}
	\item M/M/s
	\item M/D/1
	\item M/G/5
	\item M/M/10/9
	\item M/M/20/$\infty$/LIFO
\end{itemize}
\subsection{Modelos}
Tras haber introducido la notación de Kendall, es imposible evitar notar
que existen más de 30 combinaciones posibles para definir modelos distintos.
Sin embargo, no todos los modelos han sido estudiados de manera formal
a lo largo de la historia, sino únicamente un pequeño subconjunto de estos.
En este documento no se presentarán todos los modelos estudiados formalmente por 
la teoría de colas, sino únicamente los 3 modelos de relevancia para la solución 
del problema estudiado en este documento. \\
Por cada modelo se presentan únicamente las fórmulas relevantes
para la solución del problema en cuestión,
deducidas por estos investigadores para su fácil utilización
en la práctica. La demostración de estas fórmulas escapa el alcance de este documento
y tal como fué expuesto con anterioridad, estas pruebas 
involucran un gran conocimiento de matemática discreta, series, sucesiones, 
probabilidad, estadísitica, entre otras disciplinas para su comprensión. Si 
el lector desea evaluar estas pruebas, puede referirse al libro:
A First Course in Stochastic Models, Wiley, Chichester, 2003.
\subsubsection{M/M/s}
Como lo indica su notación, este modelo indica un proceso de llegadas de Poisson,
un tiempo entre servicios exponencial, una cantidad de servidores paralelos 
$s$, una población infinita, una disciplina FIFO y una capacidad de cola infinita. \\
Este modelo es uno de los más comunes en la teoría de colas y en consecuencia ha 
sido rigurosamente estudiado a lo largo de la historia. 
\begin{Large}
\begin{equation}
	\label{eq7:p0}
	P_{0} = 
\frac{1}{\frac{\left(\frac{\lambda}{\mu}\right)^{s} \sum_{n=s}^{\infty} \left(\frac{\lambda}{\mu s}\right)^{n - s}}{s!} + \sum_{n=1}^{s - 1} \frac{\left(\frac{\lambda}{\mu}\right)^{n}}{n!} + 1} 
\end{equation}\\

\begin{equation}
	\label{eq8:pn}
	P_{n} = 
\begin{cases} 
\frac{P_{0} \left(\frac{\lambda}{\mu}\right)^{n}}{n!} & n \leq s \\\frac{P_{0} s^{- n + s} \left(\frac{\lambda}{\mu}\right)^{n}}{s!} & n > s
\end{cases}
\end{equation}\\

\begin{equation}
	\label{eq13:lqmms}
	\frac{P_{0} \rho \left(\frac{\lambda}{\mu}\right)^{s}}{\left(1 - \rho\right)^{2} s!}
\end{equation}\\
\end{Large}
\subsection{M/D/s}
Como lo indica su notación, este modelo indica un proceso de llegadas de Poisson,
un tiempo entre servicios degenerado (constante), una cantidad de servidores paralelos 
$s$, una población infinita, una disciplina FIFO y una capacidad de cola infinita. \\
\begin{Large}
\begin{equation}
	\label{eq10:lqmds}
	L_{q} = \frac{\rho^{2}}{2 - 2 \rho}
\end{equation}
\end{Large}
\subsection{$M/E_{k}/1$}
Como lo indica su notación, este modelo indica un proceso de llegadas de Poisson,
un tiempo entre servicios con distribución de Erlang y un solo servidor. \\
El modelo Degenerado supone una varianza 0 en los tiempos de servicio (por lo tanto
los tiempos de servicio son constantes), mientras que la distribución exponencial
supone una gran varianza $\sigma^{2} = \frac{1}{\mu}$. Entre estos dos extremos,
hay un gran intervalo $\left(0 < \sigma < \frac{1}{\mu} \right)$, donde caen la
mayor parte de las distribuciones de tiempos de servicio reales. Una distribución 
teórica de tiempos de servicio que concuerda con este espacio intermedio es 
la Distribución de Erlang, llamada así en honor al fundador de la teoría de colas.\\
Como se ha mencionado antes, la demostración de las fórmulas presentadas a continuación
escapa el alcance de este documento. También es importante notar que solamente 
las fórmulas que son de relevancia para el problema actual serán presentadas.

\begin{Large}
\begin{equation}
	\label{eq14:media}
	Media = \frac{1}{\mu}
\end{equation}
\end{Large}

\begin{Large}
\begin{equation}
	\label{eq15:desviacion}
	\sigma = \frac{1}{\mu \sqrt{k}}
\end{equation}
\end{Large}
\begin{Large}
\begin{equation}
	\label{eq12:llq}
	L_{q} = \frac{1+k}{2 k} \frac{\lambda^{2}}{\mu \left( \mu - \lambda \right)}
\end{equation}
\end{Large}
\subsection{Redes de Colas}
Se le llama una red de colas a 2 o más colas conectadas en serie. Esto es muy común 
en la práctica y por tanto las redes de colas han sido rigurosamente estudiadas a
lo largo de la historia. Sin embargo, este es un campo difícil y por tanto 
solo se estudiará una parte muy pequeña pertinente al presente problema de estudio. \\
La propiedad más importante de las redes de colas y la única que se cubrirá en este
documento es la Propiedad de Equivalencia.
\subsubsection{Propiedad de Equivalencia}
Suponga que una instalación de servicio tiene s servidores, un proceso de entradas
de Poisson con parámetro $\lambda$ y la misma distribución de los tiempos 
de servicio de cada servidor con parámetro $\mu$ (Es decir un modelo M/M/s), donde 
tal modelo no se encuentra sobrecargado ($\rho < 1$). Entonces, la salida en 
estado estable de esta instalación de servicio es también un proceso de Poisson
con parámetro $\lambda$.\\\\
Es importante notar que la tasa de salida es independiente de $\mu$ siempre
que el sistema no se sobrecargue. La implicación más importante de esta propiedad 
es que estas unidades tienen que pasar a otra estación para continuar su servicio, 
esta segunda estación también tendrá entradas de Poisson y en consecuencia, pueden 
ser evaluadas por separado. 

\section{Marco Práctico}
\subsection{Enunciado del caso de estudio}
Jim Wells, vicepresidente de manufactura en Northern Airplane Company, está alterado. Su caminata de esta mañana por la
planta más importante de la compañía lo dejó de mal humor.
Sin embargo, ahora puede descargar su temperamento en Jerry
Carstairs, el gerente de producción de la planta, a quien acaba
de llamar a su oficina.
``Jerry, acabo de regresar de recorrer la planta y estoy muy
enojado'' ``Cuál es el problema, Jim'' ``Bueno, sabes cuánto
he recalcado la necesidad de disminuir nuestro inventario en
proceso'' ``Sí, hemos trabajado duro en eso'' responde Jerry.
``Pues no lo sufi ciente'' Jim sube más la voz. ``Sabes lo que
encontré junto al proceso'' ``No'' ``Cinco hojas de metal esperando 
que las formen en secciones de alas. Después, exactamente en el 
área siguiente de inspección, ¡13 secciones de alas!
El inspector estaba revisando una de ellas, pero las otras 12 sólo
estaban ahí. Tú sabes que tenemos un par de cientos de miles
de dólares atados en cada una de esas secciones de ala, es decir,
tenemos algunos millones de dólares en metal terriblemente
costoso sólo en espera. ¡No podemos tener eso''
El consternado Jerry Carstairs intenta responder. ``Sí, Jim,
estoy consciente de que la estación de inspección es un cuello
de botella. Casi nunca está tan mal como la encontraste esta mañana''
``Eso espero, claro'' replica Jim, ``pero necesitas prevenir
que algo tan perjudicial ocurra ni siquiera en ocasiones. ¿Qué
propones hacer al respecto'' Es notorio que ahora Jerry se anima 
al responder. ``Bueno, en realidad, ya he estado trabajando
en este problema; tengo un par de propuestas en la mesa y ya
pedí a un analista de IO, que es parte de mi personal, que estu-
die estas propuestas y regrese con recomendaciones'' ``Bien''
dice Jim, ``me da gusto ver que tratas de resolverlo. Dale a esto
tu prioridad más alta y repórtame cuanto antes'' ``Está bien''
promete Jerry.

El siguiente es el problema que Jerry y su analista de IO
estudian. Cada una de 10 prensas idénticas se usa para formar
las secciones de alas a partir de hojas de metal con un procesado
especial. Las hojas llegan de manera aleatoria al grupo de prensas
con una tasa media de 7 por hora. El tiempo que requiere la
prensa para formar una sección de ala a partir de la hoja de metal
tiene distribución exponencial con media de 1 hora. Cuando termina,
la sección de ala llega de manera aleatoria a una estación
de inspección con la misma tasa media que las hojas llegan a las
prensas (7 por hora). Un solo inspector tiene el trabajo de tiempo
completo de inspeccionar las secciones de ala para asegurar
que cumplen con las especifi caciones. Cada inspección toma 7
y1/2 minutos, de manera que puede inspeccionar 8 secciones
por hora. Esta tasa de inspección ha dado como resultado una
cantidad promedio sustancial de inventario en proceso en la
estación de inspección (esto es, el número promedio de secciones
de ala que esperan pasar la inspección es bastante grande),
además del que se encuentra en el grupo de máquinas.
El costo de este inventario en proceso se estima en \$8 por
hora por cada hoja de metal en las prensas o cada sección de ala
en la estación de inspección. Por lo tanto, Jerry Castairs tiene
dos propuestas alternativas para reducir el nivel de inventario
en proceso.
La propuesta 1 es usar un poco menos de potencia en las
prensas (lo que aumentaría su tiempo promedio para formar
una sección de ala a 1.2 horas), de manera que el inspector
pudiera mantener el paso con su producción. Esta medida tam-
bién reduciría el costo de energía para operar las máquinas
de \$7.00 a \$6.50 por hora. (Por el contrario, incrementar la
potencia al máximo aumentaría este costo a \$7.50 por hora y
disminuiría el tiempo promedio para formar cada sección de
ala a 0.8 horas.)
La propuesta 2 es sustituir el inspector por uno más joven para
esta tarea. Éste es un poco más rápido (aunque con alguna variabilidad
en sus tiempos de inspección porque tiene menos experiencia), por
lo que tendría un mejor paso. (Su tiempo de inspección tendrá una
distribución de Erlang con media de 7.2 minutos y parámetro de 
forma k = 2.) Este inspector se encuentra en una clasificación
de puestos que requiere una compensación total (incluyendo prestaciones)
de \$19 por hora, mientras que el inspector actual tiene una 
clasificación más baja y gana \$17 por hora. (Los tiempos de 
inspección de cada inspector son los normales en esa categoría en 
la clasificación). \\\\
Usted es el analista de IO en el 
equipo de Jerry Castairs a quien le han pedido que estudie este
problema. Él quiere ``que utilice las últimas técnicas
en IO para ver cuánto disminuiría cada propuesta el 
inventario en proceso para hacer sus recomendaciones''
\\\\Enunciado tomado en su totalidad de \cite[p.~800]{libroIO2}
\subsection{Análisis del estado actual}
Comenzaremos el estudio evaluando la situación actual de la empresa.
\subsubsection{Modelo}
\begin{itemize}
	\item Fase1
		\begin{itemize}
			\item \large{Notación: } \textbf{$M/M/10$}
			\item \large{Proceso de entrada: } \textbf{Poisson}
			\item \large{Tiempo de servicio: } \textbf{Exponencial}
			\item \large{Tasa media de llegada: } \textbf{7 alas/hora}
			\item \large{Tasa media de servicio: } \textbf{1 alas/hora}
			\item \large{Servidores: } \textbf{10}
		\end{itemize}
	\item Fase2
		\begin{itemize}
			\item \large{Notación: } \textbf{$M/D/1$}
			\item \large{Proceso de entrada: } \textbf{Poisson}
			\item \large{Tiempo de servicio: } \textbf{Constante (Degenerado)}
			\item \large{Tasa media de llegada: } \textbf{7 alas/hora}
			\item \large{Tasa media de servicio: } \textbf{8 alas/hora}
			\item \large{Servidores: } \textbf{1}
		\end{itemize}
\end{itemize}

\subsubsection{Costo Total}
Para poder evaluar el estado actual del sistema y poder compararlo con las propuestas
posteriores es necesario calcular el costo total por hora en que se incurre utilizando
el modelo actual, el cual se calcula de la siguiente forma: \\
$ CostoTotal = \left(L_{1}+L_{2}\right)*8+7+17$ \\\\
\subsection{Inventario en proceso}
El inventario en proceso se refiere a la cantidad de hojas de metal en el sistema 
fase1 (prensas) ya sea en la cola o en servicio denotada por $L_{s1}$ o simplemente $L_{1}$
más la cantidad de secciones de ala en el sistema fase2 (inspección) denotado como
$L_{2}$. Debido a que el resto de valores de interés son constantes, bastará 
con calcular dichos valores.
\subsubsection{Fase 1}
Para el modelo M/M/s se tiene: \\\\
\begin{large}
$\lambda = 7$\\
$\mu = 1$\\
$ s = 10$\\
\end{large}
Sustituyendo en \ref{eq7:p0} se obtiene: $P_{0}=\frac{155520}{181993351}$\\\\
Sustituyendo en \ref{eq1:lsDef} se obtiene: $L = 7.51737283706224$

\subsubsection{Fase 2}
Se sabe que el proceso de entrada para la fase 2 es un proceso de Poisson con media 7 
debido a la Propiedad de Equivalencia (Ver marco teórico).\\
Por tanto, para el modelo M/D/s se tiene: \\\\
\begin{Large}
$\lambda = 7$\\
$\mu = 8$\\
$s = 1$\\
\end{Large}
Sustituyendo en \ref{eq9:rho} se obtiene: $\rho = 0.875$ \\
Sustituyendo en \ref{eq10:lqmds} se obtiene: $L_{q} = 3.0625$\\
Sustituyendo en \ref{eq6:llq} se obtiene: $L = 3.9375$
\subsubsection{Total}
\noindent $CostoTotal = \left( L_{1} + L_{2} \right) 8 + 7 + 17$ \\
$CostoTotal = \left( 7.51737283706224 + 3.9375 \right) 8 + 7 + 17$ \\
$CostoTotal = 11.45487283706224*8 + 7 + 17$\\
$CostoTotal = 115.64 $\\
El sistema actual gasta un total de \$115.64 por hora para producir e inspeccionar
los segmentos de ala.

\subsection{Propuesta 1}
La propuesta uno consiste en bajar la potencia de las prensas para aumentar el 
tiempo de servicio medio a 1.2 horas disminuyendo de esta forma la tasa 
de servicio media ($\mu$). La idea de esta estrategia es ahorrar 
en el costo de energía y al mismo tiempo intentar bajar la tasa de llegadas a la fase 
de 2 con el objetivo de tener menos unidades en cola en la fase 2.
\subsubsection{Modelo}
\begin{itemize}
	\item Fase1
		\begin{itemize}
			\item \large{Notación: } \textbf{$M/M/10$}
			\item \large{Proceso de entrada: } \textbf{Poisson}
			\item \large{Tiempo de servicio: } \textbf{Exponencial}
			\item \large{Tasa media de llegada: } \textbf{7 alas/hora}
			\item \large{Tasa media de servicio: } \textbf{0.833 alas/hora}
			\item \large{Servidores: } \textbf{10}
		\end{itemize}
	\item Fase2
		\begin{itemize}
			\item \large{Notación: } \textbf{$M/D/1$}
			\item \large{Proceso de entrada: } \textbf{Poisson}
			\item \large{Tiempo de servicio: } \textbf{Constante (Degenerado)}
			\item \large{Tasa media de llegada: } \textbf{7 alas/hora}
			\item \large{Tasa media de servicio: } \textbf{8 alas/hora}
			\item \large{Servidores: } \textbf{1}
		\end{itemize}
\end{itemize}

\subsubsection{Fase 1}
Para el modelo M/M/s se tiene: \\\\
\begin{large}
$\lambda = 7$\\
$\mu = 0.833$\\
$ s = 10$\\
\end{large}
Sustituyendo en \ref{eq7:p0} se obtiene: $P_{0}= 0.000166659753221446$ \\\\
Sustituyendo en \ref{eq1:lsDef} se obtiene: $L = 11.0488436584526$

\subsubsection{Fase 2}
Se sabe que el proceso de entrada para la fase 2 es un proceso de Poisson con media 7 
debido a la Propiedad de Equivalencia (Ver marco teórico). Esta propiedad también 
establece que, media vez el sistema no se encuentre sobrecargado (es decir que 
el factor de utilizacion $\rho$ es menor que 1), entonces la tasa de llegada media
es igual a la tasa de salida. En el caso de esta propuesta la idea es bajar 
la tasa media de servicio con el objetivo de intentar bajar la tasa de salida 
(y en consecuencia la tasa de llegada a la siguiente fase). Sin embargo, esta 
estrategia no funcionará, ya que al bajar $\mu$ a 0.833 en consecuencia del aumento 
en los tiempos de servicio, el sistema fase 1 efectivamente será mas lento y por 
tanto aumentará la cantidad de unidades en cola. Sin embargo este cambio aún 
no sobrecarga el sistema y por tanto no afectará la tasa de salida. En otras palabras, 
esta estrategia no cumplirá su objetivo inmediato de bajar la tasa media de llegadas
a la fase 2 y de hecho tendrá un efecto 
contraproducente al mantener la tasa de llegadas a la fase 2 e incrementar la 
cantidad de hojas de metal en cola de la fase 1.
%La prueba de la Propiedad 
%de Equivalencia en la que se basa esta conclusión 
%está por fuera del alcance de este documento, sin embargo con el objetivo de probarle 
%al lector que esta propuesta tendrá un efecto contraproducente para el sistema y 
%mantendrá la tasa media de salidas, se provee una prueba empírica utilizando 
%software de simulación (Simio) para el experimento. Esta prueba empírica puede
%encontrarse en los anexos de este documento. 
Los resultados presentados a continuación sobre la fase 2 
son en efecto los mismos que aquellos presentados en el análisis del estado actual 
del sistema, ya que como se ha dicho antes, la estrategia 1 no tendrá 
ningún efecto sobre la 
fase 2 y por tanto los resultados serán los mismos.\\\\
\begin{Large}
$\lambda = 7$\\
$\mu = 8$\\
$s = 1$\\
\end{Large}
Sustituyendo en \ref{eq9:rho} se obtiene: $\rho = 0.875$ \\
Sustituyendo en \ref{eq10:lqmds} se obtiene: $L_{q} = 3.0625$\\
Sustituyendo en \ref{eq6:llq} se obtiene: $L = 3.9375$
\subsubsection{Total}
\noindent $CostoTotal = \left( L_{1} + L_{2} \right) 8 + 6.5 + 17$ \\
$CostoTotal = \left( 11.0488436584526 + 3.9375\right) 8 + 6.5 + 17$ \\
$CostoTotal = 14.9863436584526*8 + 6.5 + 17$\\
$CostoTotal = 143.40 $\\
Tras implementar la propuesta 1, el sistema gastaría un total de 
\$143.40 por hora para producir e inspeccionar
los segmentos de ala. Como era de esperarse, los resultados de implementar esta
propuesta serían contraproducentes para la empresa. Estos resultados se dan debido
al aumento en el nivel de inventario en fase 1 y la conservación del nivel de 
inventario en la fase 2 (fenómeno provocado por la Propiedad de Equivalencia).
\subsection{Propuesta 2}
La propuesta dos parece tener un mucho mejor acercamiento al problema ya 
que la idea es sustituir al inspector actual por uno más veloz, con el 
objetivo de aumentar la tasa de servicio y disminuir la cantidad de segmentos 
de ala en cola. Esta estrategia sin embargo, involucra un aumento en la paga del
nuevo inspector, por lo que se procederá a evaluar su viabilidad.
\subsubsection{Modelo}
\begin{itemize}
	\item Fase1
		\begin{itemize}
			\item \large{Notación: } \textbf{$M/M/10$}
			\item \large{Proceso de entrada: } \textbf{Poisson}
			\item \large{Tiempo de servicio: } \textbf{Exponencial}
			\item \large{Tasa media de llegada: } \textbf{7 alas/hora}
			\item \large{Tasa media de servicio: } \textbf{1 alas/hora}
			\item \large{Servidores: } \textbf{10}
		\end{itemize}
	\item Fase2
		\begin{itemize}
			\item \large{Notación: } \textbf{$M/E_{2}/1$}
			\item \large{Proceso de entrada: } \textbf{Poisson}
			\item \large{Tiempo de servicio: } \textbf{Distribución de 
				Erlang con parametro k = 2.}
			\item \large{Tasa media de llegada: } \textbf{7 alas/hora}
			\item \large{Tasa media de servicio: } \textbf{8.333 alas/hora}
			\item \large{Servidores: } \textbf{1}
		\end{itemize}
\end{itemize}
\subsubsection{Fase 1}
Esta estrategia no cambia en lo absoluto la fase 1 del sistema, por lo que 
los resultados son exactamente los mismos. A manera de recordatorio,
la cantidad esperada de unidades en el sistema para esta fase es:
$ L = 7.51737283706224$
\subsubsection{Fase 2}
En base a la Propiedad de Equivalencia, para la fase 2 se tiene:
Proceso de entrada de Poisson con media 7, tiempo entre servicios 
de Erlang, 1 servidor, capacidad de cola infinita, disciplina FIFO, y población infinita.
Por tanto, para el modelo $M/E_{k}/s$ se tiene: \\\\
\begin{Large}
$\lambda = 7$\\
$\mu = 8.333$\\
$k = 2$\\
$s = 1$\\
\end{Large}
Sustituyendo en \ref{eq12:llq} se obtiene: $L_{q} = 3.30759591995092$\\
Sustituyendo en \ref{eq6:llq} se obtiene: $L = 4.14759927996436$
\subsubsection{Total}
\noindent $CostoTotal = \left( L_{1} + L_{2} \right) 8 + 7 + 19$ \\
$CostoTotal = \left( 7.51737283706224 + 4.14759927996436\right) 8 + 7 + 19$ \\
$CostoTotal = 11.6658356172229*8 + 7 + 19$\\
$CostoTotal = 119.326684937783$\\
Al implementar la propuesta 2, se espera gastar un total de \$119.32 por hora
para producir e inspeccionar los segmentos de ala. \\
Estos resultados pueden parecer
desalentadores a primera vista, ya que de hecho son ligeramente peores que los
resultados obtenidos con el modelo actual. \\
Sin embargo, es importante considerar que esta
propuesta modela el tiempo de servicio del nuevo inspector con una distribución
de Erlang con un parámetro k = 2. 
La elección de esta distribución para el nuevo inspector no fué aleatoria. Es 
importante analizar porqué se seleccionó esta distribución.\\
Primero debemos conocer porqué la distribución Degenerada (constante) se escogió
para el inspector actual. La razón es de hecho algo intuitiva: 
Debido a la alta experiencia que el inspector actual tiene en la empresa, este
ha logrado decrecer su varianza en los tiempos de servicio hasta el punto en que
esta es 0 y por tanto sus tiempos de servicio son constantes. \\
Ahora bien, este nuevo inspector aún no tiene la misma experiencia que el 
actual y por tanto se espera que tenga cierta variabilidad en sus tiempos
de servicio. Un modelo que describe tiempos de servicio con una alta variabilidad
es el modelo exponencial, con una varianza de $\sigma = 1/\mu$. Sin embargo, 
se ha investigado que el nuevo inspector no tiene una desviación estándar tan alta.
Se sabe entonces que el nuevo inspector tendrá una varianza considerable, pero 
no tan alta como la que se propone en el modelo exponencial y tampoco es 0 como en 
el modelo degenerado. Para modelar este tipo de situaciones se utiliza la 
distribución de Erlang, la cual es una distribución más general que se 
ajusta a distintas varianzas utilizando para esto el parámetro k.
Un parametro k  = 1 indica la mayor varianza posible = $\frac{1}{\mu^{2}}$
mientras que un parámetro k = $\infty$ indica una varianza de 0.
Se puede entonces utilizar k para modelar modelos con distintas varianzas, mientras
mayor sea k, menor la varianza. Por esta razón se escogió esta distribución 
y se investigó que un valor de k = 2 sería apropiado para diseñar el modelo 
inicial.\\
Ahora bien, conocer esta razón es importante porque aún si al momento de implementar
esta estrategia y contratar al nuevo inspector no se obtiene ninguna ganancia, esto
se debe a su alta variabilidad al comenzar el trabajo. Sin embargo, conforme 
el nuevo inspector pase tiempo en la empresa y adquiera experiencia, se espera
que este disminuya su variabilidad hasta llegar al punto en que consiga tanta
experiencia en el trabajo como el inspector actual, en cuyo punto su varianza 
sería 0 (o cercana a 0) y en ese punto ya se podrán obtener ganancias. 
Por esta razón, esta estrategia no representa ganancias ni inmediatas ni a corto plazo,
pero si representa ganancias a largo plazo. A continuación se adjuntan 2 gráficos 
para ilustrar este proceso.\\
\begin{figure}[h]
\begin{center}
	\includegraphics[width=250pt]{propuesta1-1.png}
		 \caption{Gráfico del costo total por hora en función de la desviación estandar}
\label{fig1:propuesta11}
\end{center}
\end{figure}	\\
\begin{figure}[h]
\begin{center}
	\includegraphics[width=250pt,keepaspectratio]{propuesta1-2.png}
		 \caption{Gráfico del costo total por hora en función del parámetro k}
\label{fig2:propuesta12}
\end{center}
\end{figure} \\
Como se observa en la figura (\ref{fig2:propuesta12}) el costo total por hora
tras implementar la estrategia 2 es de \$119.33 (k = 2) lo cual es incluso
un poco más alto
que el precio por hora actual. Sin embargo, con forme k crece (El inspector adquiere
más experiencia y disminuye su varianza) el precio por hora decrece, hasta el 
punto en que su varianza sea casi 0 (tenga tanta experiencia como el inspector 
actual) y el costo total por hora en ese punto sería de \$110.68 generando 
finalmente un pequeño ahorro en comparación al modelo actual.
\subsection{Propuesta 3}
Tras análizar el fracaso de la propuesta 1 noté que esta fallaba por el hecho
de alentizar la fase 1 y mantener el paso hacia la fase 2 en consecuencia de la
Propiedad de Equivalencia. Por tanto, si se realiza la operación opuesta: 
Acelerar el paso en la fase 1 y mantener el paso hacia la fase 2 entonces 
deben obtenerse mejores resultados. Esta propuesta analiza esta opción.
Por tanto esta propuesta aumentará al máximo la potencia de las máquinas aumentado
sus tiempos de servicio medio a 0.8 horas. Se mantendrá el inspector actual.
\subsubsection{Modelo}
\begin{itemize}
	\item Fase1
		\begin{itemize}
			\item \large{Notación: } \textbf{$M/M/10$}
			\item \large{Proceso de entrada: } \textbf{Poisson}
			\item \large{Tiempo de servicio: } \textbf{Exponencial}
			\item \large{Tasa media de llegada: } \textbf{7 alas/hora}
			\item \large{Tasa media de servicio: } \textbf{1.25 alas/hora}
			\item \large{Servidores: } \textbf{10}
		\end{itemize}
	\item Fase2
		\begin{itemize}
			\item \large{Notación: } \textbf{$M/D/1$}
			\item \large{Proceso de entrada: } \textbf{Poisson}
			\item \large{Tiempo de servicio: } \textbf{Degenerada}
			\item \large{Tasa media de llegada: } \textbf{7 alas/hora}
			\item \large{Tasa media de servicio: } \textbf{8 alas/hora}
			\item \large{Servidores: } \textbf{1}
		\end{itemize}
\end{itemize}
\subsubsection{Fase 1}
Para el modelo M/M/s se tiene: \\\\
\begin{large}
$\lambda = 7$\\
$\mu = 1.25$\\
$ s = 10$\\
\end{large}
Sustituyendo en \ref{eq7:p0} se obtiene: $P_{0}=0.00365721364027299$\\\\
Sustituyendo en \ref{eq1:lsDef} se obtiene: $L = 5.68841994534135$

\subsubsection{Fase 2}
Debido a la Propiedad de Equivalencia y al hecho de que en esta propuesta 
se conserva al inspector actual, por lo que los resultados
de la segunda fase son exactamente iguales a los del estado actual del sistema.
A modo de recordatorio se tiene: $L = 3.9375$
\subsubsection{Total}
\noindent $CostoTotal = \left( L_{1} + L_{2} \right) 8 + 7.5 + 17$ \\
$CostoTotal = \left( 5.6884199453413 + 3.9375 \right) 8 + 7.5 + 17$ \\
$CostoTotal = 9.62591994534135*8 + 7.5 + 17$\\
$CostoTotal = 101.5073595627308$\\
El costo total esperado por hora es de \$101.51. Como se puede observar, 
los resultados son mucho mejores! Los resultados inmediatos
son mucho mejores a ambas propuestas anteriores y esto se debe a la disminución 
de hojas de metal en proceso en la fase 1 y el mantenimiento de la tasa de salida
hacia la fase 2.
\subsection{Propuesta 4}
Tras el éxito de la propuesta 3 es importante considerar la estrategia que 
implementaría la misma estrategia en la fase 1 pero al mismo tiempo contrataría 
al nuevo inspector, para verificar que opción sería mas conveniente.
\subsubsection{Modelo}
\begin{itemize}
	\item Fase1
		\begin{itemize}
			\item \large{Notación: } \textbf{$M/M/10$}
			\item \large{Proceso de entrada: } \textbf{Poisson}
			\item \large{Tiempo de servicio: } \textbf{Exponencial}
			\item \large{Tasa media de llegada: } \textbf{7 alas/hora}
			\item \large{Tasa media de servicio: } \textbf{1.25 alas/hora}
			\item \large{Servidores: } \textbf{10}
		\end{itemize}
	\item Fase2
		\begin{itemize}
			\item \large{Notación: } \textbf{$M/E_{2}/1$}
			\item \large{Proceso de entrada: } \textbf{Poisson}
			\item \large{Tiempo de servicio: } \textbf{Distribución de 
				Erlang con parametro k = 2.}
			\item \large{Tasa media de llegada: } \textbf{7 alas/hora}
			\item \large{Tasa media de servicio: } \textbf{8.333 alas/hora}
			\item \large{Servidores: } \textbf{1}
		\end{itemize}
\end{itemize}
\subsubsection{Fase 1}
Esta estrategia no cambia en lo absoluto la fase 1 del sistema estudiada 
en la propuesta 3, por lo que 
los resultados son exactamente los mismos. A manera de recordatorio, se tiene:
$ L = 5.68841994534135$
\subsubsection{Fase 2}
Debido a la Propiedad de Equivalencia, los resultados de la Fase 2 
son exactamente iguales a aquellos estudiados en la 
propuesta 2 por lo que a modo de recordatorio se tiene:
$L = 4.14759927996436$
\clearpage
\subsubsection{Total}
\noindent $CostoTotal = \left( L_{1} + L_{2} \right) 8 + 7.5 + 19$ \\
$CostoTotal = \left( 5.68841994534135 + 4.14759927996436\right) 8 + 7.5 + 19$ \\
$CostoTotal = 9.83601922530571*8 + 7.5 + 19$\\
$CostoTotal = 105.18815380244568$\\
El costo total por hora esperado es de \$105.19. 
Como se puede observar, los resultados son mucho mejores a los del 
sistema actual, sin embargo son un poco peores que los resultados de la estrategia 3.
Esto se debe como se explicó con anterioridad debido a la alta variabilidad en los 
tiempos de servicio del nuevo inspector. Sin embargo, tal como se explicó antes, 
con forme el tiempo, se espera que el nuevo inspector gane experiencia (disminuya 
su varianza) y se obtengan mejores resultados. A continuación un gráfico 
del costo por hora esperado en función del parámetro k. A modo de recordatorio:\\
k = 1 representa la máxima varianza y es igual a la distribución exponencial. \\
k = 2 es el estado del modelo actual. \\
k = $\infty$ representa una varianza de 0. 
\begin{figure}[h]
\begin{center}
	\includegraphics[width=250pt,keepaspectratio]{propuesta2.png}
		 \caption{Gráfico del costo total por hora en función del parámetro k}
\label{fig3:propuesta2}
\end{center}
\end{figure}
\clearpage
\section{Resultados}
A modo de resumen, se presenta un gráfico del costo total esperado por hora
para cada una de las estrategias en función del parámetro k. \\
\begin{figure}[h]
\begin{center}
	\includegraphics[width=250pt,keepaspectratio]{resultados.png}
		 \caption{Gráfico del costo total por hora en función del parámetro k para cada una de las propuestas viables}
\label{fig4:resultados}
\end{center}
\end{figure}

\begin{table}[h]
\centering
\begin{tabular}{  | m{5.5cm} | m{3cm} | m{3cm} | }
 \hline
	Propuesta & Costo (\$) & Ahorro (\$) \\ [0.5ex] 
 \hline\hline
	Estado Actual & 115.64 & 0 \\ 
 \hline
	Propuesta 1 & 143.40 & -27.76 \\ 
 \hline
	Propuesta 2 & 119.33 & -3.69 \\ 
 \hline
	Propuesta 3 & 101.51 & 14.13 \\ 
 \hline
	Propuesta 4 & 105.19 & 10.45 \\ [1ex]
 \hline
\end{tabular}
	\caption{Costo total y ahorro esperado por hora para cada propuesta al
	momento de su implementación. Fuente: Elaboración propia, 2020}
\end{table}

\begin{table}[h]
\centering
\begin{tabular}{  | m{5.5cm} | m{3cm} | m{3cm} | }
 \hline
	Propuesta & Costo (\$) & Ahorro (\$) \\ [0.5ex] 
 \hline\hline
	Estado Actual & 115.64 & 0 \\ 
 \hline
	Propuesta 1 & 143.40 & -27.76 \\ 
 \hline
	Propuesta 2 & 109.54 & 6.1 \\ 
 \hline
	Propuesta 3 & 101.51 & 14.13 \\ 
 \hline
	Propuesta 4 & 96.42 & 19.22 \\ [1ex]
 \hline
\end{tabular}
	\caption{Costo total y ahorro esperado por hora para cada propuesta a largo plazo.
	Fuente: Elaboración propia, 2020.}
\end{table}
\clearpage
\section{Conclusiones}
\begin{itemize}
	\item Se debe implementar la propuesta 4 para obtener el máximo ahorro.
	\item El nuevo inspector tendrá una varianza relativamente alta al
		ser contratado, impactando sus resultados negativamente. Sin embargo
		con forme este adquiera más experiencia y disminuya su varianza
		se obtendrán resultados mucho mejores.
	\item Si un sistema de colas es del tipo M/M/s y 
		no se encuentra sobrecargado ($\rho < 1$)
		entonces la tasa de salidas es igual a la tasa de llegadas.
\end{itemize}
\section{Recomendaciones}
La estrategia final a implementar para obtener el máximo ahorro es la siguiente:\\\\
Contratar al nuevo inspector más jóven y reemplazar al inspector actual. 
Aumentar la potencia de las máquinas al máximo.\\\\
Estas 2 simples acciones tendrán un gran impacto en el sistema y generarán el máximo 
ahorro. Esto debido a que las máquinas serán capaces de procesar las hojas 
de metal más rápido (0.8 horas) disminuyendo así considerablemente la cantidad
de hojas de metal que se encuentran en la fase 1 de producción. Además de esto,
el nuevo inspector si bien tendrá resultados similares al inspector actual y encima
necesitará un pago mayor al momento de ser contratado, conforme este adquiera más
experiencia y disminuya su varianza el nuevo inspector obtendrá resultados 
mucho mejores al inspector actual, agregando valor a largo plazo. En cambio, 
el inspector actual ya está al máximo de sus capacidades (varianza de 0) y por 
tanto le será imposible mejorar a lo largo del tiempo y por tanto no agregará
valor a largo plazo. \\\\
El ahorro total con esta nueva estrategia se estima en 
exactamente \$19.22 por hora. Utilizando el estándar de 173.33 horas de trabajo
por mes, el ahorro total por mes se estima en \$3331.40\\
Se le recomienda a los altos mandos implementar esta estrategia lo antes posible.
\clearpage
\section{Anexos}
\begin{figure}[h]
\begin{center}
	\includegraphics[width=\textwidth,keepaspectratio]{diagrama1.png}
		 \caption{Mapa mental de los conceptos básicos de teoría de colas}
\label{fig5:mapa}
\end{center}
\end{figure}
\begin{figure}[h]
\begin{center}
	\includegraphics[width=\textwidth,keepaspectratio]{diagrama2.png}
		 \caption{Mapa mental de los modelos más comunes y estudiados 
		 de la teoría de colas}
\label{fig6:mapa}
\end{center}
\end{figure}
\begin{figure}[h]
\begin{center}
	\includegraphics[width=\textwidth,keepaspectratio]{modelo1.png}
		 \caption{Vista en 2D de la simulación del sistema utilizando Simio}
\label{fig6:mapa}
\end{center}
\end{figure}
\begin{figure}[h]
\begin{center}
	\includegraphics[width=\textwidth,keepaspectratio]{modelo2.png}
		 \caption{Vista en 3D de la simulación del sistema utilizando Simio}
\label{fig6:mapa}
\end{center}
\end{figure}

Link de la simulación funcionando: \textcolor{blue}{https://youtu.be/mR1qPBKjb48}
%\begin{large}
%	Link de la simulación funcionando: 
%	\href{https://youtu.be/mR1qPBKjb48}{https://youtu.be/mR1qPBKjb48}
%\end{large}
%~\cite{libroION2}
%~\cite{libroION3}
%~\cite{libroION4}
%Aftewards you must re-compile the bibliography like so:
%!bibtex investigacion
%Then delete the citations and re-compile pdflatex like so:
%!pdflatex -synctex=1 -interaction=nonstopmode %
\bibliographystyle{apacite}
\bibliography{proyecto2.bib}
\end{document}
