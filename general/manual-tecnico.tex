\documentclass{article} 
\usepackage[left=1cm,right=1cm,top=1cm,bottom=1cm]{geometry} 
\usepackage[spanish]{babel}
\usepackage{listings}     
\usepackage{xcolor} 
\setcounter{secnumdepth}{0}
\usepackage{titlesec}
\titleformat{\section}[block]
  {\Large\bfseries\filcenter}	
  {} 
  {1em}
  {}

%------------------------------ Constants ---------------------------------
\newcommand{\nombre}{Renato Flores}
\newcommand{\carnet}{201709244}
\newcommand{\curso}{Sistemas Operativos 1}
\newcommand{\titulo}{Proyecto \#1}
%----------------------- Custom commands ------------------------------
\newcommand*\rbreak{\par\noindent\linebreak}
%% ----- Code highliting style specification -----------------
\lstdefinestyle{customc}{
  belowcaptionskip=1\baselineskip,
  breaklines=true,
  %frame=L,
 % xleftmargin=\parindent, 
 % Use the commented options above if you'd like a left margin for the code block.
 % useful to highlight the fact that is a code snippet.
  language=C,
  showstringspaces=false,
  basicstyle=\footnotesize\ttfamily,
  keywordstyle=\bfseries\color{green!40!black},
  commentstyle=\itshape\color{purple!40!black},
  identifierstyle=\color{blue},
  stringstyle=\color{orange},
}
\lstset{style=customc} 
%% ----------------------- Actual document ------------------------------

\author{\nombre , \carnet}
\title{\titulo}	

\begin{document}
\maketitle

\section{Aplicación Cliente}
La aplicación cliente es una aplicación de terminal interactiva escrita
en python. La aplicación utiliza la libreria de python Requests para enviar las peticiones al servidor.
Esta aplicación es extremadamente minimalista tanto en su implementación como en su uso final.
Presenta 4 opciones:
\begin{enumerate}
  \item Enviar datos \\
    Recibe la ruta absoluta de un archivo de texto. El archivo
    es dividido en oraciones y estas se envian como publicaciones
    al servidor. Estas publicaciones se envian bajo la autoria del
    usuario activo. El usuario debio haber seleccionado la opcion
    de registrarse con anterioridad para ser reconocido como el 
    autor de las publicaciones. Sin un usuario envia publicaciones
    sin identificarse primero, el usuario anonimo sera identificado
    como el autor.
  \item Balanceador \\
    Recibe la direccion http del servidor al cual enviar la informacion.
    Por default se utiliza el balanceador de carga implementado por mi.
    Si se desea utilizar un balanceador o servidor diferente, ingresar
    la direccion completa en este apartado. Es importante 
    que la direccion sea un una direccion http completa del tipo:
    \textit{http://url:port/path}  Tanto el puerto como la direccion
    del recurso deben ser especificados en la direccion del balanceador.
    No utilizar simplemente la URL!
  \item Show \\
    Esta opcion muestra todos los mensajes enviados al servidor 
    en la sesion actual junto con su respectivo autor.
    Tambien muestra la direccion de servidor utilizada actualmente 
    por la aplicacion.
  \item Autor \\
    Esta opcion sirve para que el usuario pueda identificarse, 
    simplemente debes escribir tu nombre con el cual deseas 
    firmar tus publicaciones y este sera utilizado cuando 
    envies los datos. Puedes cambiar de usuario en cualquier momento.
\end{enumerate}

\section{Servidor(s) 1}
El Servidor 1 tiene como tarea funcionar de una manera similar
a un balanceador de carga cacero. El objetivo es que este 
pregunte por el estado de los dos servidores destino (A y B).
En base al estado de estos, el Servidor 1 debe escoger hacia
que servidor se debera enviar la carga.

Este servidor contiene una API REST escrita en Python con la 
ayuda de la libreria Flask. Al igual que la aplicacion 
cliente esta es muy sencilla en su implementacion, ya 
que unicamente tiene una tarea:\\
Verificar el estado de ambos servidores destino y enviar 
el paquete recibido al servidor que mas le convenga, basandose
en la cantidad de elementos insertados, CPU utilizado en ese 
momento y RAM utilizada. Esta aplicacion 
se encuentra alojada en el puerto 5000 y esta desplegada 
sobre un contenedor de Docker basico.

\section{Servidor A y B}

Estos servidores son los mas importantes del sitema entero.
Estos se encargan de almacenar las publicaciones de los 
clientes y se encargan de exponer una API REST por medio de la
cual estos pueden ser monitorizados en cualquier momento. \\
Estos servidores tienen desplegados 3 piezas de software muy importantes:
\begin{itemize}
  \item Una API REST implementada en Python con ayuda de Flask 
    utilizada para exponer metricas respecto a la utilizacion 
    de los recursos del servidor.
  \item Una base de datos noSQL utilizada para almacenar las 
    publicaciones de los clientes.
  \item 2 Modulos del Kernel personalizados para proveer 
    estadisticas de uso en cualquier momento de los recursos 
    del servidor.
\end{itemize}

La aplicacion REST es muy sencilla y basica, simplemente 
consulta el contenido de los archivos de proceso generados 
por los modulos del kernel y los expone. Tambien 
tiene endpoints utilizados para guardar y exponer publicaciones.\\\\
La base de datos noSQL implementada es MongoDB y se implemento
en la forma de un contenedor de Docker. Tanto la base datos como la 
API rest se implementaron como contenedores en vez de instalarlos 
directamente en el servidor.\\\\
\subsection{Modulos del Kernel}
Los modulos del kernel implementados especificamente 
para el monitoreo del servidor son los siguientes:
\subsubsection{Monitor-CPU}
Este modulo indica la fraccion de uso del CPU. Este uso
es agregado y se basa en el uso de todos los cores del servidor. 
(En este caso, las instancias EC2 que alojan estos servidores
solo tienen un core, pero en teoria, si el servidor tuviera 
mas nucleos, la estadistica mostrada como uso del CPU seria 
en promedio por todos los nucleos). 
La fraccion de uso del CPU que calcula este modulo se 
calcula de la siguiente manera:

$ CPU = (B(t+1) - B(t))/(T(t+1) - T(t))$ \\
$ B(t) =  user + nice + system + iowait + irq + softirq + steal $ \\
$ T(t) =  idle + user + nice + system + iowait + irq + softirq + steal $ \\

Donde: 
\begin{itemize}
  \item user: Tiempo utilizado en ejecutar procesos en modo usuario.
  \item nice: Tiempo empleado en procesos de baja prioridad ejecutandose en modo usuario.
  \item system: Tiempo empleado en procesos ejecutandose en el kernel.
  \item iowait: Tiempo en que el procesador espera una operacion de E/S.
  \item irq: Tiempo sirviendo interrupciones por hardware.
  \item softirq: Tiempo sirviendo interrupciones por software.
  \item idle: Tiempo en que el CPU ha estado holgazan. (En espera de instrucciones)
\end{itemize}
Estos valores son medidos en unidades de tiempo denominadas 
generalmente como Jiffies (Usualmente centesimas de segundo)
Y es la cantidad de tiempo dedicado a cada una de estas categorias por
el cpu desde que el sistema se ha encendido hasta el instante t. \\\\
B(t) hace referencia al tiempo total que el CPU ha sido utilizado 
desde que se encendio el sistema (Busy). T(t) hace referencia
al tiempo total que ha estado el CPU encendido, ya sea ocupado (busy)
o holgazan (idle). \\
Estos numeros no nos dicen nada por si solos, pues a nosotros 
nos interesa la fraccion de uso del CPU en un instante t.
Para lograr esto, se aplica un concepto similar al de una derivada.\\
En teoria el porcentaje de uso del CPU en un instante t es igual 
al tiempo que el cpu ha estado ocupado en ese instante sobre la cantidad
de tiempo total que el cpu ha estado encendido en ese instante. 
En teoria, es imposible obtener el porcentaje de uso del cpu en 
un instante t, ya que si el intervalo de tiempo tiende ha ser 
tan pequeno que llega a ser un diferencial de tiempo, el tiempo 
de uso y encendido en ese intervalo sera 0. \\
Por esta razon, no se utilizo un intervalo de tiempo tan 
pequeno como lo seria un micro-segundo. Aun si el Kernel acepta 
tiempo de espera tan pequenos. Si se hubiera escogido un intervalo 
de tiempo tan pequeno para calcular el porcentaje de uso del CPU, 
este numero hubiera sido demasiado volatil e impredecible que 
no aportaria ninguna informacion util. El intervalo de tiempo
que finalmente se escogio para el calculo del porcentaje de uso 
en un instante t, es de 0.5 segundos.

Ahora bien, debido a que C no maneja de la mejor manera el formateo 
de numeros de punto flotante cuando estos alcanzan dimensiones 
tan inmensas como unsigned long long, se ha tenido que 
realizar el calculo empleado metodos de aritmetica.

La forma en que se han logrado dividir dos numeros del tipo ull (Unsigned 
long long) Ha sido de la siguiente manera:

\begin{lstlisting}
r = 1; //contador
a = r_sum[1] - r_sum[0];
b = r_idle[1] - r_idle[0];
c = 0; //c representa la cantidad de 0s necesarios que le lleva a a b.
while(a < b){
//Debemos aplicar metodos aritmeticos para calcular
//El cociente de a/b. Ya que b es siempre >> a y por tanto se 
//trunca a 0 por default.
    a = b * 10;
    c++;
}

a = a * 1000; //We add extra precision

result = a/b; //En este punto a >> b, probocando una division entera
//truncada, sin embargo la parte entera en este punto es realmente
//la parte decimal de b/a. c se utiliza para agregar los 0s 
//que le lleva b a a.

  seq_putc(p,'0');
  seq_putc(p,'.');
    while(r < c){
        seq_putc(p,'0');
       r++;
    }

\end{lstlisting} 

\subsubsection{Monitor-RAM}
El monitoreo de la memoria RAM es un poco mas sencilla. Pues 
el Kernel de Linux trae varias librerias disenadas especificamente
para dicha tarea. Si bien el calculo de la memoria RAM utilizada
por el sistema no es trivial si se hace a mano, el Kernel 
implementa un Struct denominado sysinfo. Este struct y sus 
respectivos headers proveen una funcion de utilidad denominada
si\_mem\_available(). La cual calcula un estimado de la cantidad de
memoria RAM disponible para alojar nuevos procesos sin permitir
que el sistema entre a Swap. Este calculo se realiza en base
a varias metricas como el tamanio del stack, heap, segmento de codigo
cargados en memoria, paginas de espacio de datos compartidas, cache,
buffers y varias otras metricas propias del kernel. 
El modulo implementado por mi simplemente utiliza este valor para
calcular el porcentaje de memoria utilizado en un instante t, de 
la forma:
$RAM = (totalMem - availableMem)/totalMem$
Debido a que estos tambien son numeros muy grandes (unsigned long)
La division de punto flotante tambien se implemento a mano 
siguiendo el algoritmo utilizado por el modulo del CPU.
\section{Servidor Web}
Finalmente se implemento un servidor web implementado con el 
framework React que permite al usuario monitorear los servidores
A y B en tiempo real desde el navegador.
Es importante notar que el monitor web NO implementa el uso de sockets
u otro tipo de tecnologias bi-direccionales para monitoreo de 
aplicaciones. Sino utiliza un modelo polling mucho mas basico y simple
que se basa en realizar peticiones a cada servidor durante un intervalo
de tiempo corto y constante.
Las publicaciones se consultan a un intervalo base de 3 segundos.
Las metricas de monitoreo del servidor como el uso de RAM, CPU y la
cantidad de elementos almacenados se realiza cada 5 segundos.
La aplicacion en si es muy sencilla, pues consta unicamente de 2 partes:
\begin{itemize}
  \item Vista de publicaciones de cada servidor
  \item Monitoreo de ambos servidores
\end{itemize}
Finalmente, es importante aclarar que el servidor web 
despliega el cliente web en 2 puertos: El puerto 80 con la ayuda
de Apache server, y el puerto 8080 con la ayuda de un 
contenedor de Docker corriendo sobre una maquina Alpine linux 
sirviendo la aplicacion con Node's http-server. La aplicacion web
servida en ambos puertos es exactamente la misma y es una pagina 
web estatica, esta se obtiene del proceso de compilacion 
de la aplicacion escrita en React hacia html,css y JS puros que 
pueden ser servidos por cualquier servidor web (en este caso se
emplearon tanto Apache como http-server en Docker).

\section{Apagado de instancias y manejo de direcciones IP}
Para evitar cargos extra, todas las instancias EC2 que alojan 
estos servidores (5 servidores) han sido apagados. Esto sin embargo, 
implica que la siguiente vez que se enciendan, tendran una 
direccion IP distinta y un nombre de DNS distinto. Estas 
direcciones sin embargo estan quemadas en el codigo fuente de las 
aplicaciones desplegadas en los servidores. 

Para solucionar este problema, se emplea el uso de bash 
scripts que actualizan el codigo de los servidores de manera
automatica con las nuevas
direcciones. Dichas direcciones se obtienen directamente 
de AWS utilizando la AWS CLI. 
Una vez estos se encuentren actualizados, se envian 
a los servidores por medio del protocolo SCP. Posteriormente 
se utiliza ssh para enviar el comando de actualizacion a cad instancia.
Esto elimina la necesidad de conectarse a cada instancia interactivamente
a mano y realizar las actualizaciones. Las unicas instancias a las 
que se necesita conectarse interactivamente son los servidores A \& B, 
ya que la carga de los modulos del kernel debe hacerse con el usuario 
root y por motivos de seguridad no se puede conectar a ninguna instancia 
EC2 directamente como root. Sin embargo, el deployment de estos 
modulos puede hacerse facilmente con 2 comandos: sudo su y mount.sh\\
A continuacion se muestra el script de deployment empleado 
para automatizar el despliegue del codigo fuente a los servidores.

\begin{lstlisting}[bash]
#!/bin/sh

cd ~/aws/sopes
echo "Actualizando ambos servidores 1."

rawA="http://$(./get-address servidorA):5000"
rawB="http://$(./get-address servidorB):5000"

A="'$rawA'"
B="'$rawB'"


# Servidor1(s). Debemos cambiar las direcciones de A & B

S1="$(cat ~/python/balancer/app.py)"

echo "$S1" | awk '
{ if(NR == 5) print "S_A = '"$A"'";
else if (NR == 6) print "S_B = '"$B"'"; 
else print $0 }' > ~/python/balancer/app.py 

./send-file servidor1-1 /home/renato/python/balancer
./send-file extra /home/renato/python/balancer true

./command servidor1-1 'cd balancer; sh update'
./command extra 'cd balancer; sh update'

echo "Actualizando el servidor WEB":

BUFFER="$(cat ~/python/web/src/demo.js)"
A="const pollURL = '$rawA/statements';"
B="const pollURL = '$rawB/statements';"

echo "$BUFFER" | awk '
{ if(NR == 9) print "'"$A"'"; else print $0 }' > ~/python/web/src/demo.js 

BUFFER="$(cat ~/python/web/src/demoB.js)"
echo "$BUFFER" | awk '
{ if(NR == 9) print "'"$B"'"; else print $0 }' > ~/python/web/src/demoB.js 

A="const pollA = '$rawA/inspect';"
B="const pollB = '$rawB/inspect';"

BUFFER="$(cat ~/python/web/src/monitor.js)"
echo "$BUFFER" | awk '
{ if(NR == 6) print "'"$A"'"; else if(NR == 7) print "'"$B"'"; else print $0 }' > ~/python/web/src/monitor.js 

cd ~/python/web
npm run build
cd ~/aws/sopes

./send-file web /home/renato/python/web/build
./send-file web /home/renato/python/web/update
./send-file web /home/renato/python/web/update-apache
./send-file web /home/renato/python/web/Dockerfile

./command web 'sh update'
./command web 'sh update-apache'

echo "Actualizando los servidores A & B."

./send-file servidorA /home/renato/python/monitor/cpu
./send-file servidorA /home/renato/python/monitor/ram

./send-file servidorB /home/renato/python/monitor/cpu
./send-file servidorB /home/renato/python/monitor/ram

./send-file servidorA /home/renato/python/compose
./send-file servidorB /home/renato/python/compose

./command servidorA 'cd compose && sh update'
./command servidorB 'cd compose && sh update'

echo "La Stack entera ha sido actualizada.
Conectando a los servidores A & B manualmente,
ya que los modulos del kernel deben ser re-subidos unicamente 
por el usuario root"

./connect servidorA
./connect servidorB

echo "Deployment successful"
\end{lstlisting} 
\end{document}
