%\documentclass{article} %Use titlepage option to set title in a page of its own
\documentclass[a4paper,12pt]{article}
\usepackage[left=0.5cm,right=0.5cm,top=2cm,bottom=3cm]{geometry} %Use geometry package to set custom margin sizes
\usepackage{graphicx} %Needed to insert images!
%\usepackage[section]{placeins} %Needed to ensure images don't fall out of their sections.
\usepackage[spanish]{babel}	%Use babel package to set \setcounter{secnumdepth}{0}	%Change the number if you need numbering on the sections.

\usepackage{fancyhdr} %Use for footers, can also be used for headers.
\pagestyle{fancy}
\fancyhf{}
\fancyhead[LE,RO]{201709244}
\fancyhead[RE,LO]{Renato Josue Flores Perez}
\fancyfoot[LE,RO]{Arquitectura de Computadores y Ensambladores 2, Ing. Gabriel Diaz}

\renewcommand{\headrulewidth}{2pt}
\renewcommand{\footrulewidth}{1pt}

\usepackage{titlesec}	%Use titlesec package to customize the formatting of titles
\titleformat{\section}[block]
  {\Large\bfseries\filcenter}	%Use Huge for even bigger headers
  {} %Add thesection within the braces if you want to display header number.
  {1em}
  {}

%------------------------------ Constants ---------------------------------
\newcommand{\titulo}{Aplicacion de reportes en Processing}
%----------------------- Custom commands ------------------------------
\newcommand*\rbreak{\par\noindent\linebreak}
%% ----- Code highliting style specification -----------------
\graphicspath{{/home/renato/screenshots/arqui/}}

\begin{document}
\section{\titulo}
\begin{center}
\textbf{\today}
\end{center}
La interfaz de usuario de la aplicación esta diseñada con simplicidad 
y amigabilidad en mente. Esta diseñada para ser sumamente intuitiva
y fácilmente navegable. 
Esta presenta 1 Select Box, el cual el usuario puede utilizar para 
seleccionar el reporte que desea visualizar. Tras seleccionar un 
reporte, aparecerá otro Select Box sobre el cual el usuario 
podrá seleccionar la fecha o mes de interés según aplique al reporte.
La implementacion de Select Boxes sobre Text Boxes ayuda al usuario a
ubicar fácilmente lo que necesita y previene errores.
A continuación se muestra un ejemplo de cada uno de los 
reportes disponibles.
\begin{figure}[h]
	\includegraphics[width=\textwidth,keepaspectratio]{rep1.png}
		 \caption{Reporte de pesos de paquetes entregados por día}
\end{figure}
\pagebreak
\begin{figure}[h]
	\includegraphics[width=\textwidth,keepaspectratio]{rep2.png}
		 \caption{Reporte de paquetes entregados por dia en un mes}
\end{figure}
\pagebreak
\begin{figure}[h]
	\includegraphics[width=\textwidth,keepaspectratio]{rep3.png}
		 \caption{Reporte de obstaculos encontrados por entregas en un dia}
\end{figure}
\pagebreak
\begin{figure}[h]
	\includegraphics[width=\textwidth,keepaspectratio]{rep4.png}
		 \caption{Reporte de tiempos de ida y vuelta por entregas en un dia}
\end{figure}
\pagebreak
\begin{figure}[h]
	\includegraphics[width=\textwidth,keepaspectratio]{rep5.png}
		 \caption{Reporte de tiempos promedio de ida y vuelta por dia en un mes}
\end{figure}
\end{document}
