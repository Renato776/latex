\documentclass[titlepage]{article} %Use titlepage option to set title in a page of its own
\usepackage{blindtext}	%Use blind text to produce random text
\usepackage[left=2cm,right=2cm,top=2cm,bottom=2cm]{geometry} %Use geometry package to set custom margin sizes
\usepackage{graphicx} %Needed to insert images!
\usepackage[spanish]{babel}	%Use babel package to set language for the document.

\setcounter{tocdepth}{3} 	%Set to 5 if you need even further numbering.
% paragrahs and subparagraphs can be numbered too!
\setcounter{secnumdepth}{3}	%Same as above, set to 5 if further numbering is needed.

\usepackage{titlesec}	%Use titlesec package to customize the formatting of titles
\titleformat{\section}[block]
  {\Large\bfseries\filcenter}	%Use Huge for even bigger headers
  {} %Add \thesection within the braces if you want to display header number.
  {1em}
  {}
%----------------------------- Constants ------------------------
\newcommand{\nombre}{Renato Flores}
\newcommand{\carnet}{201709244}
\newcommand{\universidad}{Universidad de San Carlos de Guatemala}
\newcommand{\catedratico}{Ing. José Saquimux}
\newcommand{\curso}{Matemática Aplicada 2}
\newcommand{\titulo}{Proyecto \#1}
% ---------------------- Custom comands -----------------------------
\newcommand*\rbreak{\par\noindent\linebreak} %My own custom linebreaks!

\begin{document}

\begin{titlepage} 
	\centering
	\includegraphics[width=0.15\textwidth]{/home/renato/latex/caratula/logo.png}\par\vspace{1cm}
	{\scshape\LARGE \universidad \par}
	\vspace{1cm}
	{\scshape\Large \curso \par}
	\vspace{1.5cm}
	{\huge\bfseries \titulo \par}
	\vspace{2cm}
	{\Large \nombre , \carnet \par}
	\vfill
	catedrático\par
	\catedratico

	\vfill

% Bottom of the page
	{\large \today\par}
\end{titlepage}

\tableofcontents
\clearpage

\section{Introducción}
Las Google Cloud APIs son una gran parte de la plataforma cloud de Google.
Con ayuda de estas APIs puedes facilmente agregar el poder de todos 
los servicios que provee google con tus aplicaciones, esto incluye 
tanto los servicios comunes utilizados por millones de usuarios al 
rededor del mundo como lo son Gmail, Calendar, Drive, Docs, etc. Como
tambien los servicios menos famosos diseñados especificamente para su 
uso programatico como machine-learning-based APIs, Business Analytics,
entre otros. \\
Las APIs de Google son gratis para su uso personal, y tambien tienen 
la opcion de integrarse con una empresa por medio del uso de la GSuite 
de dicha empresa. \\
Puedes accesar a estas APIs desde cualquier aplicacion que soporta 
las librerias cliente de Google. Entre ellas se encuentran:
\begin{itemize}
  \item Android
  \item NodeJS
  \item Python
  \item Go
  \item Java
  \item PHP
\end{itemize}
\section {APIs REST}
Las APIs de Google siguen el modelo REST. 
Una API REST es un modo sencillo de acceder a servicios web sin excesivo procesado. Cuando se llama a una API RESTful, el servidor transferirá al cliente una representación del estado de recurso solicitado.
En realidad, esto lo hacemos casi cada día. Si está buscando vídeos en YouTube, teclearás una palabra clave en su buscador, pulsarás Enter y obtendrás un listado de vídeos. Conceptualmente, así es como funciona una API REST. Estás buscando algo y obtienes una lista de resultados del servicio al que has lanzado la petición.
Una API es un interfaz de programación de aplicaciones. Es un conjunto de reglas que permite a los programas comunicarse entre ellos. El desarrollador crea la API en el servidor y permite al cliente ‘hablar’ con ella.
REST es lo que determina el ‘aspecto’ de la API. Son las reglas que siguen los desarrolladores cuando crean una API. Una de estas reglas determina que deberías ser capaz de obtener un ‘trozo’ de datos (un recurso), cuando invocas una URL determinada.
Cada llamada a una URL se denomina una petición, mientras que los datos obtenidos se denomina repuesta.

\subsection{Metodos de de una API REST}
HTTP tiene cinco métodos que se usan normalmente en arquitecturas basadas en REST: POST, GET, PUT, PATCH y DELETE. En realidad se corresponden con crear, leer, actualizar y borrar, respectivamente. Conviene recordar que existen también otros métodos que se usan con menos frecuencia, como OPTIONS y HEAD.
\subsubsection{GET}
Este método se usa para recuperar (o leer) una representación de un recurso. Si todo va bien, GET devuelve una representación en XML o JSON y en código HTTP de respuesta 200 (OK). En caso de error, habitualmente devuelve un 404 (no encontrado) o 400 (petición incorrecta).
\subsubsection{POST}
 Este método se usa a menudo para crear nuevos recursos. En particular, recursos subordinados, es decir subordinado a algún otro recurso (padre). Al crearse con éxito, devuelve un estado HTTP 201, devolviendo una cabecera con el enlace al recurso recién creado.
 \subsubsection{PUT}
 Se usa para actualizar y también crear un recurso (en el caso en el que el identificador del recurso lo elige el cliente en lugar del servidor). Esencialmente, PUT se lanza a una URL que contiene el valor de un recurso no existente. Una actualización con éxito devuelve un 200 (o 204 si no se devuelve ningún contenido). Si se usa PUT para crear, devuelve un HTTP 201 si se crea con éxito.
 \subsubsection{PATCH}
 Se usa para modificar capacidades. La petición PATCH sólo necesita los cambios del recurso, no el recurso completo. Es parecido a PUT, sin embargo el cuerpo contiene un conjunto de instrucciones que describen cómo un recurso que actualmente reside en el servidor debe ser modificado para generar una versión. Así pues, el cuerpo del PATCH no debería ser sólo la parte modificada del recurso, sino algún tipo de lenguaje como XML o JSON.
 \subsubsection{DELETE}
 Muy explicativo, se usa para borrar recursos identificados por una URL. Al borrar con éxito, devuelve un estado HTTP 200 (OK), junto con un cuerpo de respuesta.
\subsection {Respuesta}
Luego de realizar una llamada exitosa a alguna API de Google por medio
de cualquiera de sus metodos, la mayoria de sus APIs retornan una respuesta JSON que generalmente sigue el siguiente esquema:
%Insert image here: practicas-1.png
Donde, la informacion que realmente nos interesa se encuentra en 
el objeto data.
El status \& status test, por lo general son siempre 200, "OK". Ya que 
si en caso ocurriera un error o nuestra peticion fuera rechazada por Google, retorna un HTTP error code correspondiente y la estructura de la respuesta cambia totalmente.
\subsection{Partes de una respuesta}
Si bien se explico antes sobre la forma en que se recibe una respuesta 
estandar de Google, a continuacion se explica mas detalladamente, 
las partes de la respuesta.
\subsubsection{Config}
Contiene metadata sobre el tipo de solicitud, el servidor, la version y 
otra informacion sobre el tipo de solicitud realizada.
\subsubsection{Data}
La informacion que solicitamos.
\subsubsection{Headers}
Headers estandar de una respuesta HTTP.
\subsubsection{Status}
Codigo de status de la solicitud siempre sera 200 si esta es exitosa.
\subsubsection{StatusText}
Una pequeña descripcion del status de la respuesta, siempre sera "OK"
si esta es exitosa.
\subsubsection{Request} Indica el resource path utilizado para realizar
esta request.
\section{Autenticacion}
Si bien es cierto que las APIs de Google pueden ser accesadas desde 
basicamente cualquier dispositivo con una conexion a internet, su consumo en realidad no es tan simple. 
TODAS las apis de Google requieren de una autenticacion por 
parte del usuario para utilizarlas.
Los procesos de autenticacion disponibles incluyen.
\subsection{API Keys}
Una API Key es la forma mas simple de autenticacion, ya que 
unicamente requieren que estas se incluyan como una QueryString 
en la request. Esto implica que cualquier dispositivo puede emplearlas, 
en teoria es posible utilizarlas directamente desde Arduino o cualquier dispositivo con conexion a internet.\\
Sin embargo, no son perfectas y tienen varias desventajas.
A continuacion se listan ambas.
\subsubsection{Ventajas}
\begin{itemize}
  \item Facil de generar
  \item Facil de utilizar
  \item Se pueden utilizar desde cualquier cliente
\end{itemize}
\subsubsection{Desventajas}
\begin{itemize}
  \item Solo pueden accesar la informacion publica
  \item Solo pueden realizar operaciones de solo lectura
  \item No todas las APIs admiten autenticacion por API Keys.
\end{itemize}
\subsection{OAuth2 personal authentication}
El flujo general a seguir cuando se utiliza la verificacion 
personal por medio de OAuth2 tokens es:
\begin{enumerate}
  \item Crear un proyecto de desarrollador de Google
  \item Habilitar las APIs a utilizar por el proyecto
  \item Crear un proyecto local en el lenguaje de eleccion.
  \item Descargar e instalar las librerias para autenticacion con OAuth2
  \item Descargar las credenciales de autenticacion de la consola de desarrolladores de Google
  \item Utilizar la libreria de OAuth2 y las credenciales descargadas para comenzar el proceso de autenticacion
  \item Aprobar el acceso de sus recursos a la aplicacion
  \item Utilizar la API.
    \subsection{Service Accounts}
    La autenticacion por Service Accounts es muy similar a la 
    autenticacion personal, sin embargo con este metodo en vez de utilizar
    credenciales personales para acceder directamente a los recursos 
    de una cuenta, se emplean las credenciales de una cuenta especial
    creada especialmente para acceder a los recursos de un dominio entero
    comprado bajo una GSuite. Estas cuentas tienen direcciones de correo
    que pueden ser utilizadas para otorgar acceso a ciertos recursos
    como si se tratara de simples cuentas personales. Sin embargo, estas
    cuentas no tienen acceso a Gmail y no tienen contraseña, pues
    no estan dirigidas al uso por humanos. La idea es utilizarlas para 
    autenticarse con Google a favor de un dominio entero y poder utilizar
    los recursos de dicho dominio. Estas cuentas
    se crean en la consola de desarrolladores de Google, y necesita 
    de permisos elevados del administrador del dominio para ser creadas.
    Esta es la forma en la que las empresas integran los servicios de 
    Google con sus aplicaciones o portales existentes. Un ejemplo de 
    ello es nuestro dominio de la facultad de ingenieria.
  \section{Scopes}
    El Scope o Alcance en español, indica de forma especifica 
    a que recurso (API) se tiene acceso y a que permitos (leer, editar)
    sobre los recursos de una cuenta se tiene acceso. Un Scope
    se indica como un URL y se incluye en el cuerpo de la autenticacion
    inicial con Google. Ejemplos de Scopes validos son:
    \begin{itemize}
      \item https://www.googleapis.com/auth/drive.readonly
      \item https://www.googleapis.com/auth/spreadsheets.readonly
      \item https://www.googleapis.com/auth/drive.file
      \item https://www.googleapis.com/auth/spreadsheets
      \item https://www.googleapis.com/auth/admin.directory.resource.calendar
    \end{itemize}
    Puede visitar el link: https://developers.google.com/identity/protocols/oauth2/scopes 
    para una lista de todos los Scopes de todas las APIs disponibles de Google.
    \section{Google APIs Library}
    De manera general, Google provee una libreria demoniada "googleapis"
    que engloba todas las APIs disponibles por google. Estas deben ser
    instanciadas en el codigo de la aplicacion cliente y se deben 
    crear objetos por cada API especifica que se desee utilizar.
    El flujo de trabajo general, es el siguiente:
    \begin{enumerate}
      \item Concectarse y Autenticarse
      \item Crear el objeto cliente que representa la API
      \item Identificar el recurso al que se desea acceder
      \item Utilizar el metodo especifico
      \item Enviar el body esperado de acuerdo al metodo
      \item Enviar un callback
      \item Procesar la informacion devuelta
    \end{enumerate}
    \section(Estructura de una Instancia de una Google API}
    De manera general, un objeto instanciado de manera correcta
    que representa una Google API luego de haber sido autenticado 
    se ve de la siguiente forma:
    %Insert image here, practicas 2
    La forma en que se agrupan los recuros y sus respectivas
    funciones puede encontrarse en la documentacion oficial
    de la API especifica de interes. \\
    Es importante notar, que el Metodo no se refiere a un 
    metodo REST estandar. La lista de metodos soportados 
    por cada recurso puede encontrarse en la documentacion oficial.
\end{document}
