\documentclass{article}
\usepackage[left=4cm,right=4cm,top=3cm,bottom=3cm]{geometry} 
\usepackage{graphicx} 
%\usepackage[section]{placeins} 
\usepackage[spanish]{babel}

%------------------------------ Constants ---------------------------------
\newcommand{\nombre}{Renato Flores}
\newcommand{\carnet}{201709244}
\newcommand{\titulo}{Ensayo}
%----------------------- Custom commands ------------------------------
\author{\nombre , \carnet}
\title{\textbf{\Huge\titulo}}

\begin{document}
\maketitle
\section {Integridad y corrupción en el rol de estudiantes}
La integridad se refiere a la recteza moral de un 
individuo. Por recteza me refiero al hecho de seguir todas las 
reglas morales y éticas definidas por la sociedad.
Ahora bien, la corrupción se refiere al hecho de quebrar o sobrepasar
el orden y funcionamiento de un sistema tanto 
ética como funcionalmente, para beneficio personal. \\\\

En el contexto de un estudiante, primero debemos definir  a
que sistema se hace referencia cuando se habla de corrupción. 
Dicho sistema es aquel definido en el Reglamento Estudiantil. \\\\

Dicho reglamento, entre otras cosas,  indica muy claramente todas las 
reglas a las que debe someterse un estudiante. Un estudiante íntegro, 
es entonces aquel que cumple en su totalidad con tadas estas reglas.
En contraste, un estudiante corrupto, es aquel que sobrepasa 
una o mas de estas reglas con el objetivo de cumplir algun deseo 
personal. \\\\

Generalmente estos deseos suelen asociarse con el hecho de
ganar algún curso o la obtención de créditos. Un estudiante corrupto
es capaz de idear artimañas muy elaboradas para sobrepasar las 
reglas, cumplir su objetivo y mantenerse anónimo. \\\\

Debido al cambio de contexto en la educación actual gracias a la 
pandemia la ejecución de ciertas artimañas se ha facilitado 
grandemente. Provocando que el estudiante se incline hacia la corrupción
debido a la facilidad de lograr sus objetivos. \\
Esta corrupción tiene 2 formas principales de combatirse: \\
\begin{itemize} 
        \item Educando al estudiante de modo que este elija no sobrepasar las reglas.
        \item Removiendo en la medida de lo posible, cualquier posibilidad
                que el estudiante tenga de realizar artimañas que 
                incumplan en cualquier medida con el 
                Normativo Estudiantil.
\end{itemize}
Sin lugar a dudas, el segundo método es el más efectivo y el escogido 
por las autoridades. Dicho objetivo se logra empleando métodos 
de viligancia y control.  \\\\
Sin embargo, ahora que la modalidad de estudio ha cambiado, los métodos
de vigilancia y control empleados con anterioridad para evitar la 
corrupción no son aplicables ni efectivos. 
Es imposible vigilar a todos los estudiantes y someterlos a procesos
de control sin interferir en su aprendizaje, ademas que tomaría una 
extrema cantidad de esfuerzo por parte de las autoridades. \\\\
En vista de esto, el método de combate contra la corrupción del 
estudiante debe cambiarse y centralizar los esfuerzos en educar 
al estudiante de forma que este elija no recurrir a artimañas sucias 
para lograr sus objetivos.

\section {¿La competencia es siempre buena?}
Si. Todas las personas tienen maneras distintas de abordar
la competencia, sin embargo para la gran mayoría de personas, 
la competencia
es una gran fuerza de motivación mucho más efectiva que otros métodos
basados en recompensas como pueden serlo el dinero, 
puntos en una asignatura, créditos, etc. Si bien esas recompensas 
son efectivas en la mayoría de ocasiones, su efectividad depende 
de las condiciones en que se encuentra la persona a la que se 
le ofrece dicha recompensa. \\\\
A modo de ejemplo, supongamos una persona por nombre Amanda. 
Si a Amanda se le ofrece dinero para que ella realice un proyecto, y 
su situación financiera es extremadamente delicada, la paga por su 
proyecto sera la principal fuente de motivación para que ella 
realice un excelente proyecto. \\\\
Ahora bien, si la situación financiera de Amanda esta bajo control 
e incluso tiene comodidades, ella aceptará el proyecto pero
la motivación de la paga no tendrá el mismo efecto sobre ella. \\\\
Este ejemplo presenta dos escenarios un tanto extremos en los que 
se evidencia como varía la efectividad de ese tipo de recompensas. 
Ahora bien, existen otro tipo de motivaciones mucho mas efectivas como 
por ejemplo la realización personal, la busqueda por nuevas experiencias,
la competencia, la busqueda por conocimiento, etc. \\\\
Ese tipo de recompensas, si bien no aplican a todas las personas, 
no dependen de la condición actual de la persona, sino de su forma 
de ser. Por tanto, cuando estas aplican conllevan una gran motivación
mucho mas efectiva que las recompensas.\\\\

En conclusión, la competencia, en la mayoría de ocasiones, 
es una gran fuerza de motivación que provoca una búsqueda constante 
por mejoría y superación.

\end{document}
