\documentclass[titlepage]{article} %Use titlepage option to set title in a page of its own
\usepackage{blindtext}	%Use blind text to produce random text
\usepackage[left=2cm,right=2cm,top=2cm,bottom=2cm]{geometry} %Use geometry package to set custom margin sizes
\usepackage{graphicx} %Needed to insert images!
\usepackage[spanish]{babel}	%Use babel package to set language for the document.

\setcounter{tocdepth}{3} 	%Set to 5 if you need even further numbering.
% paragrahs and subparagraphs can be numbered too!
\setcounter{secnumdepth}{3}	%Same as above, set to 5 if further numbering is needed.

\usepackage{titlesec}	%Use titlesec package to customize the formatting of titles
\titleformat{\section}[block]
  {\Large\bfseries\filcenter}	%Use Huge for even bigger headers
  {} %Add \thesection within the braces if you want to display header number.
  {1em}
  {}
%----------------------------- Constants ------------------------
\newcommand{\nombre}{Renato Flores}
\newcommand{\carnet}{201709244}
\newcommand{\universidad}{Universidad de San Carlos de Guatemala}
\newcommand{\catedratico}{Ing. José Saquimux}
\newcommand{\curso}{Matemática Aplicada 2}
\newcommand{\titulo}{Proyecto \#1}
% ---------------------- Custom comands -----------------------------
\newcommand*\rbreak{\par\noindent\linebreak} %My own custom linebreaks!

\begin{document}

\begin{titlepage} 
	\centering
	\includegraphics[width=0.15\textwidth]{/home/renato/latex/caratula/logo.png}\par\vspace{1cm}
	{\scshape\LARGE \universidad \par}
	\vspace{1cm}
	{\scshape\Large \curso \par}
	\vspace{1.5cm}
	{\huge\bfseries \titulo \par}
	\vspace{2cm}
	{\Large \nombre , \carnet \par}
	\vfill
	catedrático\par
	\catedratico

	\vfill

% Bottom of the page
	{\large \today\par}
\end{titlepage}

\tableofcontents
\clearpage

\section{Modulación por ancho de pulsos}
        La modulación por ancho de pulsos también abreviada PWM por sus siglas en inglés (Pulse Width Modulation) de 
        una señal o fuente de energía es una técnica en la que una onda periódica generalmente rectangular con un período T, 
        se enciende durante un cierto tiempo m dentro de su período. A este tiempo m, generalmente se le conoce como tiempo de
        encendido. A la relación entre m y T se le conoce como ciclo de trabajo (d) y se define entonces como:
        $ d = \frac{m}{T}$ En ingeniería es generalmente mas conveniente expresar estas ondas en terminos de d, la frecuencia (f) 
        y la amplitud de la onda A. Uno de sus usos más importantes es en controladores de cargadores solares los cuales se encargan 
        de prevenir que la batería se encuentre sobre cargada o totalmente descargada, ambos escenarios siendo desastrosos para la batería.
	\subsection{A test subsection}
		\blindtext
		\subsubsection{A test subsubsection}
			\blindtext \rbreak 
			\blindtext \rbreak
			\blindtext
		\subsection{Another test subsection}
			\blindtext
	\section{Another test section}
		\blindtext
	\section{A test section}
		\blindtext
	\section{Another test section}
		\blindtext
		\subsection{A test subsection}
			\blindtext
		\subsection{Another test subsection}
			\blindtext
\end{document}
