\documentclass{article} %Use titlepage option to set title in a page of its own
\usepackage{blindtext}	%Use blind text to produce random text
\usepackage[left=1cm,right=1cm,top=1cm,bottom=1cm]{geometry} %Use geometry package to set custom margin sizes
\usepackage{graphicx} %Needed to insert images!
\usepackage[section]{placeins} %Needed to ensure images don't fall out of their sections.
\usepackage[spanish]{babel}	%Use babel package to set language for the document.
\usepackage{listings}       % Needed to highlight code snippets. Amazing package!
\usepackage{xcolor} %Needed to actual highliting of source code.
%If not imported, source code will be formatted accordingly but no color syntax will be performed.
\setcounter{secnumdepth}{0}	%Change the number if you need numbering on the sections.

\usepackage{titlesec}	%Use titlesec package to customize the formatting of titles
\titleformat{\section}[block]
  {\Large\bfseries\filcenter}	%Use Huge for even bigger headers
  {} %Add thesection within the braces if you want to display header number.
  {1em}
  {}

%------------------------------ Constants ---------------------------------
\newcommand{\nombre}{Renato Flores}
\newcommand{\carnet}{201709244}
\newcommand{\universidad}{USAC}
\newcommand{\catedratico}{Ing. Otto Salazar}
\newcommand{\curso}{Auditoria y Revision de Redes 1}
\newcommand{\titulo}{Tarea - 1}
%----------------------- Custom commands ------------------------------
\newcommand*\rbreak{\par\noindent\linebreak}
%% ----- Code highliting style specification -----------------
\lstdefinestyle{customc}{
  belowcaptionskip=1\baselineskip,
  breaklines=true,
  %frame=L,
 % xleftmargin=\parindent, 
 % Use the commented options above if you'd like a left margin for the code block.
 % useful to highlight the fact that is a code snippet.
  language=C,
  showstringspaces=false,
  basicstyle=\footnotesize\ttfamily,
  keywordstyle=\bfseries\color{green!40!black},
  commentstyle=\itshape\color{purple!40!black},
  identifierstyle=\color{blue},
  stringstyle=\color{orange},
}
\lstset{style=customc} %The original manual also had escapechar=@, as a parameter.
%Not sure what it does tho. Also, make sure to include this line to set the 
%listings style globally and not needing to set it manually for each block!
%You could tho.
%% ------------------------------ Actual document ------------------------------

\author{\nombre , \carnet}
\title{\titulo}	

\graphicspath{{/home/renato/screenshots/renato/second-semester-2020/redes-1/agosto-09/}}

\begin{document}
\maketitle

\section{Instalacion de NG-EVE - Reloaded}
	%--% Wrap text in %--%  Delimiters in order to automatically escape special symbols!.
	%Special Symbols escaped by this block are: _, &, #.
	%If you need to escape other characters as well like: 
	%,\,{,} Use estrict plain text blocks defined like:
	%%__% plain text %__%% 
	both, printed & escaped. Hoping for the best!!\\It does seem to work 100\%!
	\blindtext
	\begin{figure}[h]
		\includegraphics[width=\textwidth,keepaspectratio]{part1}
		 \caption{Comprobando la conexión entre las computadoras cliente y sus respectivas interfaces del router}
	    \label{fig:conn1}
	\end{figure}
	\begin{figure}[h]
		\includegraphics[width=\textwidth,keepaspectratio]{part2}
		\caption{Comprobando la conexión entre ambas computadoras cliente}
	\end{figure}
	\FloatBarrier %Use float barrier explicitly if you 
		\subsection{Nodes do not change}
			\blindtext
	%--% Finish the plain text block.
			\begin{lstlisting}
#include <stdio.h>
#define N 10
/* Instructions 
	* Performing a node test: 
		replace text
 * The listings package can also include directly from source code!
 * You must set explictly how the syntax highlithing should look like.
 * You can do it for each codeblock, or you can write highliting configuration 
 * and export it to a style, then you can use the style whenever you call a 
 * listings block.
*/


int main()
{
    int i;

    // Line comment.
    puts("Hello world!");

    for (i = 0; i < N; i++)
    {
        puts("LaTeX is also great for programmers!");
    }

    return 0;
}
		\end{lstlisting} 
		%The listings package also supports importing directly from source code!
\end{document}