\documentclass{article} %Use titlepage option to set title in a page of its own
\usepackage{blindtext}	%Use blind text to produce random text
\usepackage[left=1cm,right=1cm,top=1cm,bottom=1cm]{geometry} %Use geometry package to set custom margin sizes
\usepackage{graphicx} %Needed to insert images!
\usepackage[section]{placeins} %Needed to ensure images don't fall out of their sections.
\usepackage[spanish]{babel}	%Use babel package to set language for the document.

\setcounter{secnumdepth}{0}	%Change the number if you need numbering on the sections.

\usepackage{titlesec}	%Use titlesec package to customize the formatting of titles
\titleformat{\section}[block]
  {\Large\bfseries\filcenter}	%Use Huge for even bigger headers
  {} %Add thesection within the braces if you want to display header number.
  {1em}
  {}

\newcommand*\rbreak{\par\noindent\linebreak}

%% ----------------- Custom support for highlithing code snippets--------------
%Its usage is appropiate if you plan on showcasing code snippets and you'd like
%to highlight an specific subset of the code. Otherwise you'd probably better of 
%using the lstlisting package.
\usepackage{verbatimbox,xcolor,lipsum}
\usepackage{verbatimbox}
\def\codesize{\footnotesize}
\newsavebox\thecolorfield
\newcommand\setcolorfield[1][blue!19!gray!18]{%
  \savebox{\thecolorfield}{\codesize%
    \makebox[0pt][l]{\textcolor{#1}{%
    \rule[-\dp\strutbox]{\textwidth}{\dimexpr\ht\strutbox+\dp\strutbox}}}}%
}

\def\colorfield{\usebox{\thecolorfield}}

\setcolorfield

\newcommand*\ifline[4]{%
  \ifnum\value{VerbboxLineNo}<#1#4\else
    \ifnum\value{VerbboxLineNo}>#2#4\else#3\fi
  \fi
}

\newcommand\rednum{\makebox[0ex][r]{\color{red}\arabic{VerbboxLineNo}:}\hspace{1ex}}

\newcommand\blacknum{\makebox[0ex][r]{\arabic{VerbboxLineNo}:}\hspace{1ex}}

%% ------------------------------ Actual document ------------------------------

\author{Renato Flores, 201709244}
\title{Tarea - 1}	

\graphicspath{{/home/renato/screenshots/renato/second-semester-2020/redes-1/agosto-09/}}

\begin{document}
\maketitle

\section{Instalacion de NG-EVE}
	\blindtext
	\begin{figure}[h]
		\includegraphics[width=\textwidth,keepaspectratio]{part1}
		 \caption{Comprobando la conexión entre las computadoras cliente y sus respectivas interfaces del router}
	    \label{fig:conn1}
	\end{figure}
	\begin{figure}[h]
		\includegraphics[width=\textwidth,keepaspectratio]{part2}
		\caption{Comprobando la conexión entre ambas computadoras cliente}
	\end{figure}
	\FloatBarrier %Use float barrier explicitly if you 
		\subsection{Nodes deployment}
			\blindtext
			\begin{verbnobox%The empty comment line is required!
			}[\codesize\ifline{8}{18}{\colorfield\rednum}{\blacknum}\ttfamily]
#include <stdio.h>
#define N 10
/* Instructions */
//ifline{8}{18} indicates subrange of code lines to draw attention to.
//It takes 4 arguments, the first 2 are the range (inclusive) of the area to highlight.
//The other 2 are format parameters to customize the appearence of highlithed area and
//non highlited area. The 3th argument formats the highlighted area, the 4th argument
//Formats the non-highlighted area. If you don't need to highlight an specific range of 
//code lines, avoid using the ifline command.

int main()
{
    int i;

    // Line comment.
    puts("Hello world!");

    for (i = 0; i < N; i++)
    {
        puts("LaTeX is also great for programmers!");
    }

    return 0;
}
\end{verbnobox}
		
\end{document}