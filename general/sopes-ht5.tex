\documentclass{article} 
\usepackage[left=1cm,right=1cm,top=1cm,bottom=1cm]{geometry} 
\usepackage[spanish]{babel}
\usepackage{listings}     
\usepackage{xcolor} 
\setcounter{secnumdepth}{0}
\usepackage{titlesec}
\titleformat{\section}[block]
  {\Large\bfseries\filcenter}	
  {} 
  {1em}
  {}

%------------------------------ Constants ---------------------------------
\newcommand{\nombre}{Renato Flores}
\newcommand{\carnet}{201709244}
\newcommand{\curso}{Sistemas Operativos 1}
\newcommand{\titulo}{Hoja de Trabajo, Semaforos y Mensajes}
%----------------------- Custom commands ------------------------------
\newcommand*\rbreak{\par\noindent\linebreak}
%% ----- Code highliting style specification -----------------
\lstdefinestyle{customc}{
  belowcaptionskip=1\baselineskip,
  breaklines=true,
  %frame=L,
 % xleftmargin=\parindent, 
 % Use the commented options above if you'd like a left margin for the code block.
 % useful to highlight the fact that is a code snippet.
  language=C,
  showstringspaces=false,
  basicstyle=\footnotesize\ttfamily,
  keywordstyle=\bfseries\color{green!40!black},
  commentstyle=\itshape\color{purple!40!black},
  identifierstyle=\color{blue},
  stringstyle=\color{orange},
}
\lstset{style=customc} 
%% ----------------------- Actual document ------------------------------

\author{\nombre , \carnet}
\title{\titulo}	

\begin{document}
\maketitle

\section{Semaforos}

\subsection{Problema de Baile de Salón}
\begin{lstlisting}
Semaforo S1,S2,S3,S4 = 0;

Hombre(){
  up(S1); down(S2);
  Bailar();
  down(S3); up(S4);
}

Mujer(){
  down(S1); up(S2);
  Baliar();
  up(S3); down(S4);
}

\end{lstlisting} 

\subsection{Babuinos}
\begin{lstlisting}
 Semaforo S,recurso = 1;
 int cuentaRec = 0;
 
 On(sem,recurso,cuentaRec){
    down(sem);
    cuentaRec = cuentaRec + 1
    if ( cuentaRec == 1) down(recurso);
    up (sem);
 }

 Off(sem,recurso,cuentaRec){
    down(sem);
    cuentaRec = cuentaRec - 1;
    if ( cuentaRec == 0) up(recurso);
    up (sem);
 }

 Babuino(babuino,sem,recurso,cuentaRec){
    down(babuino);
    On(sem,recurso,cuentaRec);
    CruzaBabuino();
    Off(sem,recurso,cuentaRec);
    up(babuino);
 }

 Semaforo semIda, semVuelta = 1;
 Semaforo babIda = 5, babVuelta = 5;
 Semaforo rec = 1;
 int cuentaIda, cuentaVuelta = 0;
 BabuinoIda(){
   Babuino(babIda,semIda,rec,ceuntaIda);
 }
 BabuinoVuelta(){
    Babuino(babViene, semVuelta,rec,cuentaVuelta);
 }

\end{lstlisting} 
\end{document}
