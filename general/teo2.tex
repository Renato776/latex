\documentclass{article}
\usepackage[left=2cm,right=2cm,top=2cm,bottom=2cm]{geometry} 
\usepackage{graphicx} 
%\usepackage[section]{placeins} 
\usepackage[spanish]{babel}

%------------------------------ Constants ---------------------------------
\newcommand{\nombre}{Renato Flores}
\newcommand{\carnet}{201709244}
\newcommand{\universidad}{USAC}
\newcommand{\catedratico}{Ing. Jorge Luis Álvarez}
\newcommand{\curso}{Teoría de Sistemas 2}
\newcommand{\titulo}{Método del Valor Ganado}
%----------------------- Custom commands ------------------------------
\author{\nombre , \carnet}
\title{\textbf{\Huge\titulo}}
\newcommand*\rbreak{\par\noindent\linebreak}
\graphicspath{{/home/renato/screenshots/teo2/}}
\begin{document}
\maketitle
La Gestión del Valor Ganado (GVG) también abreviado EVM por sus siglas en inglés es una técnica 
que proporciona un enfoque para medir el desempeño del proyecto a partir de la comparación de su avance 
real frente al planeado, permitiendo evaluar tendencias para formular pronósticos.\rbreak
Para implementar la GVG en un proyecto es necesario definir la Línea Base de Medición del Desempeño 
)Performance Measurement Baseline, PMB) que integra el alcance (EDT), el cronograma (Gantt) y el cálculo de sus costos 
y recursos requeridos para su ejecución (Presupuesto).
\section{Valores escenciales}
\subsection{Valor Planificado}
También abreviado PV (Planned Value) Es el valor de la PMB al día de la fecha.
\subsection{Valor Ganado}
Earned Value (EV) representa lo que ya se ha realizado al dia de la fecha, valuado 
con los costos usados para definir la PMB.
\subsection{Costo Real}
Actual Cost (AC) representa el costo que ha insumido el trabajo realizado hasta la fecha.
Se pueden expresar en porcentajes dividiéndolos por el Presupuesto hasta la Conclusión (Budget at Completition, BAC) \rbreak
\bfseries{Ejemplo}\rbreak
\begin{itemize}
                \item PV\% = PV/BAC
                \item EV\% = EV/BAC
                \item AC\% = AC/BAC    
\end{itemize}
                \section{Variaciones}
                \subsection{Variación del Cronograma}
                (Schedule Variance, SV). SV = EV – PV
                \subsection{Variación del Costo}
                (Cost Variance, CV). CV = EV – AC\rbreak
                \bfseries{SV\%} = SV / PV \rbreak
                \bfseries{CV\%} = CV / EV
                \section{Índices de Rendimiento}
    \subsection{Índice de Rendimiento del Cronograma} (Schedule Performance Index, SPI). SPI = EV / PV
    \subsection{Índice de Rendimiento del Costo} (Cost Performance Index, CPI). CPI = EV /AC
    \subsection{Índice del Rendimiento hasta Concluir} (To Complete Performance Index, TCPI). TCPI = (BAC – EV) / (BAC – AC).
\section{Valores de Proyección a la finalización del Proyecto}
    \subsection{Estimado a la Conclusión} (Estimate at Completion, EAC). Es el pronóstico del costo final. Puede calcularse de diferentes formas:
    \begin{itemize}
        \item EAC = BAC – SV. Los costos futuros no serán los mismos que los considerados en la PMB debido a que las variaciones del costo fueron atípicas.
        \item EAC = BAC / CPI. Los costos futuros se calcularán de acuerdo con el índice de eficiencia del rendimiento del costo a la fecha.
        \item EAC = BAC / (CPI * SPI). Los costos futuros se calcularán con base a los índices de rendimiento del costo y del cronograma a la fecha.
        \item EAC = AC + Nuevo estimado para el trabajo remanente.
    \end{itemize}
    \subsection{Estimado hasta concluir} (Estimate to Complete, ETC). ETC = EAC – AC
    \subsection{Variación a la Conclusión} (Variance at Completion, VAC). VAC = BAC – EAC \rbreak
    \bfseries{VAC\%} = VAC / BAC
    \subsection{Índice de Rendimiento del Costo a la Conclusión} (Cost Performance Index at Conclusion, CPIAC). CPIAC = BAC / EAC
    \section{Utilidad}
    A lo largo de la ejecución y supervisión del proyecto, es necesario analizar el rendimiento del proyecto para poder contestar
    a la pregunta que siempre nos hacen todos los involucrados: ¿cómo va el proyecto? De la misma manera, 
    se debe revisar las tendencias, decidir qué medidas correctivas se aplicarán y determinar los pronósticos 
    para responder la pregunta más importante: ¿cómo terminará el proyecto? \rbreak
    En cada fecha de estado debe registrarse el avance de cada tarea del proyecto de acuerdo con la
    técnica de medición del valor ganado seleccionada durante la planificación; debe, además, actualizarse el trabajo 
    remanente de la tarea. De esta manera siempre se contará con información actualizada y confiable sobre el proyecto.\rbreak
    Finalmente, es importante aclarar que el principal objetivo de la GVG es proporcionar la retroalimentación correcta 
    para facilitar la toma de decisiones. La GVG por sí misma no producirá proyectos exitosos; para ello se requiere de
    un director de proyecto dispuesto a realizar el análisis necesario y a emprender acciones correctivas cuando se lo requiera.
        \section{Pasos para la implementación general de la GVG}
        A continuación se listan los pasos a seguir de manera general para implementar la GVG de manera exitosa. Sin embargo, 
        estos pueden variar de acuerdo al proyecto en cuestión.
        \subsection{Inicio}
        \begin{itemize}
    \item Definir los parámetros iniciales y las diferentes opciones de software a utilizar
    \item Definir los umbrales de calidad que se usarán para el monitoreo y el control del proyecto
        \end{itemize}
        \subsection{Planificación}
        \begin{itemize}
    \item Definir la EDT
    \item Definir la técnica de medición del valor ganado para cada tarea
    \item Definir el cronograma dinámico
    \item Asignar los recursos/costos a todas las tareas
    \item Establecer la distribución del presupuesto a lo largo del tiempo
    \item Establecer la línea base de medición del rendimiento
        \end{itemize}
    En cada uno de los pasos anteriores se deberá revisar el paso anterior y realizar actualizaciones cuando sea necesario
\subsection{Ejecución, seguimiento y control (para cada período de informes)}
    \begin{itemize}
            \item Definir la fecha de estado
            \item Registrar el avance de cada tarea de acuerdo con la técnica de medición del valor ganado elegida
            \item Actualizar el trabajo remanente de cada tarea
            \item Desarrollar el análisis de datos de la GVG
            \item Calcular o definir pronósticos
            \item Proponer acciones correctivas según sea necesario
            \item Entregar informes de desempeño
            \item Mantener la integridad de la línea base de medición del rendimiento
    \end{itemize}
    \clearpage
\center    \section{Apéndices}
        \begin{figure}[h]
                \begin{center}
                        \includegraphics[width=250,keepaspectratio]{gvg-1.jpg}
                \end{center}
                \caption{Partes de la GVG}
        \end{figure}
        \begin{figure}[h]
                \begin{center}
                        \includegraphics[width=250,keepaspectratio]{gvg-2.jpg}
                \end{center}
                \caption{Ejemplo de la medición del valor ganado con 2 períodos de medición}
        \end{figure}
%\begin{figure}[h]
%        \includegraphics[width=\textwidth,keepaspectratio]{/home/renato/screenshots/sopes-1/security-group.png}
%                 \caption{Crear grupo de seguridad de red}
%\end{figure}	
\end{document}
