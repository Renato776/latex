%\documentclass{article} %Use titlepage option to set title in a page of its own
\documentclass[a4paper,12pt]{article}
\usepackage[left=0.5cm,right=0.5cm,top=2cm,bottom=3cm]{geometry} %Use geometry package to set custom margin sizes
\usepackage{graphicx} %Needed to insert images!
%\usepackage[section]{placeins} %Needed to ensure images don't fall out of their sections.
\usepackage[spanish]{babel}	%Use babel package to set \setcounter{secnumdepth}{0}	%Change the number if you need numbering on the sections.

\usepackage{fancyhdr} %Use for footers, can also be used for headers.
\pagestyle{fancy}
\fancyhf{}
\fancyhead[LE,RO]{201709244}
\fancyhead[RE,LO]{Renato Josue Flores Perez}
\fancyfoot[LE,RO]{Redes de computadoras 1, Sección N, Ing. Pedro Pablo Hernandez Ramirez}

\renewcommand{\headrulewidth}{2pt}
\renewcommand{\footrulewidth}{1pt}

\usepackage{titlesec}	%Use titlesec package to customize the formatting of titles
\titleformat{\section}[block]
  {\Large\bfseries\filcenter}	%Use Huge for even bigger headers
  {} %Add thesection within the braces if you want to display header number.
  {1em}
  {}

%------------------------------ Constants ---------------------------------
\newcommand{\nombre}{Renato Josue Flores Perez}
\newcommand{\carnet}{201709244}
\newcommand{\universidad}{USAC}
\newcommand{\catedratico}{Ing. Otto Salazar}
\newcommand{\curso}{Redes - 1}
\newcommand{\titulo}{Tarea Practica Individual No. 2}
%----------------------- Custom commands ------------------------------
\newcommand*\rbreak{\par\noindent\linebreak}
%% ----- Code highliting style specification -----------------
\graphicspath{{/home/renato/screenshots/redes-1/}}

\begin{document}
\section{\titulo}
\begin{center}
\textbf{\today}
\end{center}
\begin{figure}[h]
	\includegraphics[width=\textwidth,keepaspectratio]{topologia.png}
		 \caption{Topologia realizada}
\end{figure}
\pagebreak
	\begin{figure}[h]
		\includegraphics[width=\textwidth,keepaspectratio]{arch-ping.png}
		 \caption{Ping desde la maquina A (Arch) hacia la maquina B (TinyCore)}
	    \label{fig:conn1}
	    \rbreak
	    \includegraphics[width=\textwidth,keepaspectratio]{tiny-core-ping.png}
		\caption{Ping desde la maquina B (TinyCore) hacia la maquina A (Arch)}
	\end{figure}
\pagebreak
	\begin{figure}[h]
		\includegraphics[width=\textwidth,keepaspectratio]{wireshark.png}
		 \caption{Captura de paquetes ICMP}
	    \label{fig:conn1}
	\end{figure}
\end{document}
