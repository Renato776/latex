\documentclass{article}
\usepackage[left=2cm,right=2cm,top=2cm,bottom=2cm]{geometry} 
\usepackage{blindtext}
\usepackage{graphicx} 
\usepackage[section]{placeins} 
\usepackage[spanish]{babel}
\usepackage{listings} 
\usepackage{xcolor} 
\setcounter{secnumdepth}{1}

%------------------------------ Constants ---------------------------------
\newcommand{\nombre}{Renato Flores}
\newcommand{\carnet}{201709244}
\newcommand{\universidad}{USAC}
\newcommand{\catedratico}{Ing. Allan Morataya}
\newcommand{\curso}{Sistemas Operativos 1}
\newcommand{\titulo}{Hoja de trabajo \#1}
%----------------------- Custom commands ------------------------------
\newcommand*\rbreak{\par\noindent\linebreak}

\author{\nombre , \carnet}
\title{\textbf{\Huge\titulo}\\Reforzamiento de conceptos}

\graphicspath{{/home/renato/screenshots/renato/second-semester-2020/sopes-1/agosto-10/}}

\begin{document}
\maketitle
\section{Explique en qué consiste un sistema operativo.}

\section{Justifique porque la arquitectura de un SO está diseñada por capas.}

\section{Defina el concepto de proceso.}

\section{Defina el concepto de proceso.}

\section{Que es la memoria MDA?}
Es un metodo denominado formalmente como Acceso Directo a la Memoria, el cual le permite a ciertos dispositivos de entrada/salida acceder directamente a la memoria principal (RAM) sin necesidad de esperar a que el CPU se desocupe para atender al dispositivo. El objetivo es aprovechar los tiempos de espera del CPU.

\section{¿En qué consiste la planificación de procesos?}

\section{¿En qué consiste un sistema operativo online?}

\section{Mencione y explique dos estados de un proceso.}

\section{Defina el concepto de proceso.}

\section{¿Cuál es la diferencia entre aborto y suspensión, en relación a los procesos de un SO?}

\end{document}
