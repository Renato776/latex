\documentclass[osajnl,onecolumn,showpacs,superscriptaddress,10pt]{revtex4-1}
\usepackage{dcolumn}% Align table columns on decimal point
\usepackage{bm}% bold math
\usepackage[spanish]{babel}
\usepackage[utf8]{inputenc}
\usepackage[T1]{fontenc}
\usepackage{makeidx}
\usepackage{graphicx}
\usepackage{subfig}
\usepackage{amsmath}
\usepackage{amsfonts}
\usepackage{amssymb}
\usepackage[pdftex]{hyperref}
\usepackage{multirow}
\usepackage{float}
\usepackage{booktabs}
\decimalpoint
%\bibliographystyle{IEEEtran}
%\bibliography{IEEEabrv,mybibfile}

\begin{document}
%Titulo
\title{Laboratorio 1: Nombre de la practica}

\author{Renato Josue Flores Perez, 201709244}
\affiliation{Facultad de Ingenieri­a, Departamento de Fi­sica, Universidad de San Carlos, Edificio T1, Ciudad Universitaria, Zona 12, Guatemala.
}
%Add more authors as needed

\date{\today}

\maketitle{}

\section{Conclusiones}

Las conclusiones son interpretaciones logicas del analisis de resultados, que
deben ser consistentes con los objetivos presentados previamente. 

\begin{enumerate}
\item conclussion 1
\item conclussion 2
\item etc.
\end{enumerate}


\end{document}